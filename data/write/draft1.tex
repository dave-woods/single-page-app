\documentclass[a4paper,11pt]{article}

\usepackage{graphicx}  %%% for including graphics
\usepackage{url}       %%% for including URLs
\usepackage{times}
\usepackage{amssymb,amsmath,amscd}
\usepackage{natbib}
\usepackage[margin=25mm]{geometry}

% Tim's custom commands
\newcommand{\bc}{{\rm b\!c}}
\newcommand{\unpad}{\mbox{{\rm unpad}}}
\newcommand{\vph}[1]{\vphantom{#1}}
\newcommand{\sta}[2]{\stackrel{#1}{#2}}
\newcommand{\sk}{{\sf k}}
\newcommand{\sv}{{\sf v}}
\newcommand{\rr}{{\rm r}}

\title{Preliminary Title}
\date{}

\author{Example Author\\
	Affiliation\\
	\texttt{example@email.org}
	\and Someone Else\\
	Another Affiliation\\
	\texttt{another@email.org}
}

\begin{document}
\maketitle
\thispagestyle{empty}
\pagestyle{empty}

%\section{Structure}
%
%\begin{itemize}
%	\item Why strings are useful $\to$ bring information together
%	\item Can't presuppose translation of Allen Relations to strings
%	\item Block compression
%	\item Superposition
%	\item Exploring usefulness of having longer strings for greater than 2 
%	events
%	\item How feasible is this representation
%	\item Trying to find the sweet spot before the combinatorial explosion
%	\item Bounded superposition approach at string level
%	\item Using a finite state automaton (looking at all strings at once)
%	\item Hybrid approach
%	\item Added constraints before superposition
%\end{itemize}

%%%%%%%%%%%%%%%%%%%%%%%%%%%%%%%%%%%%%%%%%%%%%%%%%%%%%%%%%%%%%%%%%%%%%%%%%%%%%%%%
%                                                                              %
%                            Actual Document Begins                            %
%                                                                              %
%%%%%%%%%%%%%%%%%%%%%%%%%%%%%%%%%%%%%%%%%%%%%%%%%%%%%%%%%%%%%%%%%%%%%%%%%%%%%%%%
\section{Introduction}
This paper explores using string-based models to effectively represent event 
data such as might be found in a document annotated with TimeML. It is 
described how such data may be simply translated to strings, as well as 
operations on the these resultant strings which may be used to infer new 
information.

It is put forward that strings pose a strong candidate as a basis for modelling 
this information over other approaches such as models rooted in predicate 
logic. Due to a string's nature as a computational entity, it is both simple 
and intuitive to understand as a human, and efficiently tractable for a 
machine. Strings also permit operations on an individual level, which may be 
impossible for models which generate all formulae at once.

An \textit{event-string} is a string $s \in \Sigma^+$ for an alphabet 
$\Sigma$, the powerset of some fixed set $\digamma$ of fluents \textit{i.e.} 
predicates which have an associated temporality. If the 
string $s = \alpha_1\cdots\alpha_n$ is taken as a sequence of $n$ moments in 
time, each symbol $\alpha_i \in s$ represents the set of relevant fluents 
holding at the 
moment indexed by $i$. The string is read from left to right chronologically, 
so that any predicates which hold at the moment at index $i$ are understood to 
have held before the moment indexed by $j$ if and only if $i < j$.

The precise duration of each moment is taken as unimportant in the current 
discussion, 
and thus the strings model an inertial world, whereby \textit{change} is the 
only mark of progression from one moment to the next -- ``there is no time 
without change" (Aristotle).

Thus, if $\alpha_i = \alpha_{i+1}$ for any 
$1 \leq i < n$, then either symbol may be safely deleted from $s$ without 
affecting the interpretation of the string, as the remaining symbol is simply 
taken as representing a longer moment. This operation of removing 
repetition from 
the event-string is known as \textit{block compression} 
\citep{fernando2016prior}. We may also describe 
the inverse of this process, which introduces repeated elements in an 
event-string without changing its interpretation, allowing us greater 
flexibility in our manipulation of the string. These operations are detailed 
in the next section.

\section{Superposition and Block Compression}
With two strings $s$ and $s'$ of the same length $n$
built from an alphabet $\Sigma$, the powerset of some fixed set 
$\digamma$,
the \textit{superposition} $s ~\&~ s'$ of $s$ and $s'$ is their componentwise 
union:
\begin{align*}
\alpha_1\cdots\alpha_n \ \&\ 
\alpha'_1\cdots\alpha_n' & \ :=\
(\alpha_1\cup\alpha'_1)\cdots(\alpha_n\cup\alpha'_n)
\end{align*}
For convenience of notation, we will use boxes rather than curly braces 
$\{$~$\}$ to represent sets in $\Sigma$, such that each symbol $\alpha$ in a 
string $s$ corresponds to exactly one box. For example, with $a, b, c, d \in A$:
\begin{align*}
\fbox{$a\vph{,'}$}\fbox{$c\vph{,'}$} \ \& \ 
\fbox{$b\vph{,'}$}\fbox{$d\vph{,'}$} \ = \
\fbox{$a, b\vph{,'}$}\fbox{$c, d\vph{,'}$} \ \in\ \Sigma^2
\end{align*}
Extending this function to languages $L$ and $L'$ over the same alphabet is a 
simple matter of 
collecting the superpositions of strings from each language of equal length: 
\begin{align*}
L ~\&~ L' & \ :=\ \bigcup_{n\geq 0}
\{s~\&~s'\ | \ s\in L\cap \Sigma^n\mbox{ and }s'\in L'\cap \Sigma^n\}
\end{align*}
For example, $L ~\&~ \fbox{\vph{'}}^\ast = L$.
If $L$ and $L'$ are regular languages computed by finite automata
with transitions $\to$ and $\to'$, then the superposition $L~\&~ L'$ is
a regular language computed by a finite automaton with transitions
$\Rightarrow$
formed by running $\to$ and $\to'$ in lockstep
according to the rule 
\begin{align*}
\frac{q \sta{\alpha}{\to} r   \hspace{.4in} q' \sta{\alpha'}{\to'} r'}{
	(q,q') \sta{\alpha\cup\alpha'}{\Rightarrow} (r,r')}
\end{align*}
A disadvantage of this operation is that it requires the string operands to be 
of equal length, which is an overly specific case. In order to generalise this 
procedure to strings of arbitrary lengths, we may 
manipulate the strings to introduce asynchrony. For example, we 
can cause a string $s=\alpha_1\cdots\alpha_n$ to \textit{stutter} such that 
$\alpha_i=\alpha_i+1$ for some integer $0 < i < n$. For example, 
\fbox{$a\vph{'}$}\fbox{$a\vph{'}$}\fbox{$a\vph{'}$}\fbox{$c\vph{'}$}\fbox{$c\vph{'}$}
 is a stuttering version of 
\fbox{$a\vph{'}$}\fbox{$c\vph{'}$}. If 
a string does not 
stutter, it is \textit{stutterless}, and we can transform a stuttering string 
to this 
state by using ``block compression":
\begin{align*}
\bc(s)  &\ \ :=\ \
\left\{ \begin{array}{ll}
s & \mbox{ if }~length(s)\leq 1 \\
\bc({\alpha}s')  & \mbox{ if } s={\alpha}{\alpha}s'\\
\alpha ~{\bc}({\alpha'}{s'})  
& \mbox{ if } s=\alpha{\alpha'}{s'} \mbox{ with } \alpha\neq\alpha'
\end{array}
\right. 
\end{align*}
This function can be applied multiple times to a string, but the output will 
not change after the first application: $\bc(\bc(s)) = \bc(s)$. We can also use 
the inverse of this function to generate infinitely many stuttering strings:
\begin{align*}
\bc^{-1}(\fbox{$a\vph{,'}$}\fbox{$c\vph{,'}$}) & = 
\fbox{$a\vph{,'}$}\fbox{$a\vph{,'}$}\fbox{$c\vph{,'}$} \\
& = \fbox{$a\vph{,'}$}\fbox{$c\vph{,'}$}\fbox{$c\vph{,'}$} \\
& = \fbox{$a\vph{,'}$}\fbox{$a\vph{,'}$}\fbox{$c\vph{,'}$}\fbox{$c\vph{,'}$} \\
& \hspace{3em} \vdots
\end{align*}
We can say that any of the strings generated by this inverse block compression 
are \textit{$\bc{}$-equivalent}. Precisely, a string $s'$ is $\bc{}$-equivalent 
to a 
string 
$s$ iff $s' \in \bc^{-1}\bc(s)$.

We can now define the \textit{asynchronous superposition} of strings $s$ and 
$s'$ as the (provably) \textit{finite} set obtained by block compressing the 
\textit{infinite} 
language generated by superposing the strings which are $\bc$-equivalent to $s$ 
and $s'$:
\begin{align*}
s\ \&_{\bc} \ s' & \ := \
\{\bc(s'') \ | \ s''\in \bc^{-1}\bc(s)\ \& \ \bc^{-1}\bc(s')\}
\end{align*}
For example \fbox{$a\vph{,'}$}\fbox{$c\vph{,'}$} $\&_{\bc}$ 
\fbox{$b\vph{,'}$}\fbox{$d\vph{,'}$} will comprise 
three strings:
\begin{align*}
\fbox{$a, b\vph{,'}$}\fbox{$c, d\vph{,'}$}\\
\fbox{$a, b\vph{,'}$}\fbox{$a, d\vph{,'}$}\fbox{$c, d\vph{,'}$}\\
\fbox{$a, b\vph{,'}$}\fbox{$b, c\vph{,'}$}\fbox{$c, d\vph{,'}$}\\
\end{align*}
In order to avoid generating all possible strings when using the inverse block 
compression, we introduce an upper bound to the length of the strings which 
will be superposed. It can be shown that with two strings of length $n$ and 
$n'$, the longest $\bc$-unique string (one which has no shorter 
$\bc$-equivalent strings) produced through asynchronous superposition will be 
of length $n+n'-1$.

\section{Upper Bound on Asynchronous Superposition}
\paragraph{Proposition 1.} {\sl For all  $s, s' \in \Sigma^+$
and all $s''\in s\ \hat{\&}\ s'$,}
\begin{align*}
\mbox{length}(s'')\ \leq\ \mbox{length}(s) + \mbox{length}(s') -1
\end{align*}
\noindent
For all $s,s'\in \Sigma^{\ast}$, we define a finite set
$s~\hat{\&}~s'$ of strings over $\Sigma$ with enough of 
the strings in $\bc^{-1}\bc(s)~\&~\bc^{-1}\bc(s')$
to form $s~\&_{\bc}~s'$.
The definition proceeds by induction on $s$ and $s'$, with
\begin{align*}
\epsilon\ \hat{\&}\ \epsilon \ :=& \ \{\epsilon\}\\
\epsilon\ \hat{\&}\ s \ :=& \ \emptyset\ \ \mbox{ for } s\neq\epsilon\\
s\ \hat{\&}\ \epsilon \ :=& \ \emptyset\ \ \mbox{ for } s\neq\epsilon
\end{align*}
and for all $\alpha,\alpha'\in \Sigma$,
\begin{align*}
\alpha s~\hat{\&}~\alpha's' \ :=& \ 
(\alpha\cup \alpha')(\alpha s\ \hat{\&}\ s'
\ | \ s~\hat{\&}~\alpha's' \ | \  s~\hat{\&}~s')\\
=&~\{(\alpha\cup\alpha')s''\ | \ s''\in (\alpha s~\hat{\&}~s') \cup 
(s~\hat{\&}~\alpha's') \cup (s~\hat{\&}~s')\}
\end{align*}
Note that a string in $s\ \hat{\&}\ s'$ might stutter, even if neither of the 
operands $s$ or $s'$ do
(\textit{e.g.} $\fbox{$a,c\vph{,'}$}\fbox{$a,c\vph{,'}$}\in
\fbox{$a\vph{,'}$}\fbox{$c\vph{,'}$}\ \hat{\&}\ 
\fbox{$c\vph{,'}$}\fbox{$a\vph{,'}$}$)
-- however, it can be made stutterless through block compression.

\paragraph{Proposition 2.} {\sl For all $s,s'\in \Sigma^+$,}
\begin{align*}
s\ \hat{\&}\ s' & \ \subset\ \bc^{-1}\bc(s)\ \&\ \bc^{-1}\bc(s')
\end{align*}
and
\begin{align*}
\{\bc(s'')\ | \ s''\in s\ \hat{\&}\ s'\}
& \ = \   s\ \&_{\bc}\ s'
\end{align*}
Now, for any integer $k > 0$ and string $s = \alpha_1\cdots\alpha_n$ over 
$\Sigma$, we introduce a new 
function 
$pad_k$ which will generate the set of strings with length $k$ which are 
$\bc$-equivalent to $s$:
\begin{align*}
pad_k(\alpha_1\cdots\alpha_n)~:=&~~\alpha_1^+\cdots\alpha_n^+\ \cap\ \Sigma^k \\
=&~~\{
\alpha_1^{k_1}\cdots\alpha_n^{k_n}\ | \
k_1,\ldots,k_n\geq 1
\mbox{ and } \sum_{i=1}^n k_i = k \}\\
\subset&~~\bc^{-1}\bc(\alpha_1\cdots\alpha_n)
\end{align*}
For example, $pad_{4}(\fbox{$a\vph{'}$}\fbox{$c\vph{'}$})$ will 
generate \{
\fbox{$a\vph{'}$}\fbox{$a\vph{'}$}\fbox{$a\vph{'}$}\fbox{$c\vph{'}$}, 
\fbox{$a\vph{'}$}\fbox{$a\vph{'}$}\fbox{$c\vph{'}$}\fbox{$c\vph{'}$}, 
\fbox{$a\vph{'}$}\fbox{$c\vph{'}$}\fbox{$c\vph{'}$}\fbox{$c\vph{'}$} \}. We 
can use this new function in our 
calculation of asynchronous superposition, to limit the generation of strings 
from the inverse block compression step. Since we know from Proposition 1 that 
the maximum possible 
length we might need is $n+n'-1$, we can use this value in the $pad$ function 
to just 
generate the strings of that length, giving us a new definition of asynchronous 
superposition:
\paragraph{Corollary 3.} {\sl For any $s,s'\in \Sigma^+$
	with nonzero lengths $n$ and $n'$ respectively,}
\begin{align*}
s \ \&_{\bc} \ s' & \ = \
\{\bc(s'')\ | \ s''\in pad_{n+n'-1}(s)\ \& \ pad_{n+n'-1}(s')\}
\end{align*}
Neither $s\ \hat{\&}\ s'$ nor $pad_{n+n'-1}(s)\ \& \ pad_{n+n'-1}(s')$
need be a subset of the other, even though,
under the assumptions of Corollary 3, 
both sets block compress to $s \ \&_{\bc} \ s'$.

\section{Event Representation}\label{secER}
Now we may use asynchronous superposition to generate the 13 strings in 
\fbox{\vph{$e'$}}\fbox{$e\vph{e'}$}\fbox{\vph{$e'$}} 
$\&_{\bc}$ 
\fbox{\vph{$e'$}}\fbox{$e'$}\fbox{\vph{$e'$}}, each of which corresponds to one 
of the unique interval relations in \cite{allen1983maintaining}.
We use the empty box \fbox{$\vph{'}$} as a string of length 1 (not to confused 
with the empty string $\epsilon$, which is length 0) to bound events, allowing 
us to represent the fact that they are finite -- they have a beginning and 
ending point. It is prudent to assume that we will deal only with finite event 
data, such that there are no fluents which do not have both an associated 
start-point and end-point. If such a fluent were to occur, it would be trivial 
to add it to every position in the string. 

The bounding boxes represent the time 
before and after the event occurs, during which no other fluents belonging to 
the alphabet $\Sigma$ 
hold. The 
Allen Relation Strings are laid out 
below:
\begin{center}
	\begin{tabular}{ l@{\hskip 1in}c@{\hskip 1in}r }
		$e~\mathbf{=}~e'$ & \fbox{\vph{$e'\vph{,e'}$}}\fbox{$e, 	
		e'\vph{e'}$}\fbox{\vph{$,e'$}} & equal\\[0.6em]		
		$e~\mathbf{s}~e'$ & \fbox{\vph{$e'\vph{,e'}$}}\fbox{$e, 
		e'\vph{e'}$}\fbox{$e'\vph{,e'}$}\fbox{\vph{$,e'$}} & starts\\[0.6em]
		$e~\mathbf{si}~e'$ & \fbox{\vph{$e'\vph{,e'}$}}\fbox{$e, 		
		e'\vph{e'}$}\fbox{$e\vph{,e'}$}\fbox{\vph{$,e'$}} & 		
		starts~(inverse)\\[0.6em]
		$e~\mathbf{f}~e'$ & \fbox{\vph{$e'\vph{,e'}$}}\fbox{$ 		
		e'\vph{,e'}$}\fbox{$e, e'\vph{,e'}$}\fbox{\vph{$,e'$}} & 		
		finishes\\[0.6em]
		$e~\mathbf{fi}~e'$ & \fbox{\vph{$e'\vph{,e'}$}}\fbox{$ 		
		e\vph{,e'}$}\fbox{$e, e'\vph{,e'}$}\fbox{\vph{$,e'$}} &		
		finishes~(inverse)\\[0.6em]
		$e~\mathbf{d}~e'$ & \fbox{\vph{$,e'$}}\fbox{$ e'\vph{,e'}$}\fbox{$e, 
		e'\vph{,e'}$}\fbox{$ e'\vph{,e'}$}\fbox{\vph{$,e'$}} & during\\[0.6em]
		$e~\mathbf{di}~e'$ & \fbox{\vph{$e'\vph{,e'}$}}\fbox{$ 
		e\vph{,e'}$}\fbox{$e, e'\vph{,e'}$}\fbox{$ 
		e\vph{,e'}$}\fbox{\vph{$,e'$}} & during~(inverse)\\[0.6em]
		$e~\mathbf{o}~e'$ & \fbox{\vph{$,e'$}}\fbox{$ e\vph{,e'}$}\fbox{$e, 
		e'\vph{,e'}$}\fbox{$ e'\vph{,e'}$}\fbox{\vph{$,e'$}} & 
		overlaps\\[0.6em]
		$e~\mathbf{oi}~e'$ & \fbox{\vph{$e'\vph{,e'}$}}\fbox{$ 
		e'\vph{,e'}$}\fbox{$e, e'\vph{,e'}$}\fbox{$ 
		e\vph{,e'}$}\fbox{\vph{$,e'$}} & overlaps~(inverse)\\[0.6em]
		$e~\mathbf{m}~e'$ & \fbox{\vph{$,e'$}}\fbox{$ e\vph{,e'}$}\fbox{$ 
		e'\vph{,e'}$}\fbox{\vph{$,e'$}} & meets\\[0.6em]
		$e~\mathbf{mi}~e'$ & \fbox{\vph{$e'\vph{,e'}$}}\fbox{$ 
		e'\vph{,e'}$}\fbox{$ e\vph{,e'}$}\fbox{\vph{$,e'$}} & 
		meets~(inverse)\\[0.6em]
		$e~\mathbf{<}~e'$ & \fbox{\vph{$,e'$}}\fbox{$ 
		e\vph{,e'}$}\fbox{$\vph{,e'}$}\fbox{$ e'\vph{,e'}$}\fbox{\vph{$,e'$}} & 
		before\\[0.6em]
		$e~\mathbf{>}~e'$ & \fbox{\vph{$e'\vph{,e'}$}}\fbox{$ 
		e'\vph{,e'}$}\fbox{$\vph{,e'}$}\fbox{$ e\vph{,e'}$}\fbox{\vph{$,e'$}} & 
		after
	\end{tabular}
\end{center}
These Allen Relations are included in the attributes of ISO-TimeML 
\citep{pustejovsky2010iso}, the standard 
markup language used for event annotation in texts, as TLINKs. By extracting 
the TLINKs from an annotated document, and translating them to our event-string 
representation, we may begin to reason about the relationships between 
annotated events which do not have an associated TLINK in the markup. For 
example, the document may give us a relation between events $e$ and $e'$, and 
another relation between $e'$ and $e''$, and from this we may infer the 
possible 
relations between $e$ and $e''$.

As asynchronous superposition is a commutative, associative, homomorphic binary 
relation over event-strings, we may superpose 
arbitrary numbers of event-strings: $s ~\&_{\bc}~ s' ~\&_{\bc}~ s'' ~\&_{\bc}~ 
\cdots$. Note, however, that superposing even a 
small number of unconstrained bounded events leads to a massive blowup in the 
number of possible outcomes. Currently, attempting to compute the result of 
more than five is beyond the capabilities of available hardware:
\begin{align*}
2 ~\mbox{bounded events} &\to 13 ~\mbox{outcomes}\\
3 ~\mbox{bounded events} &\to 409 ~\mbox{outcomes}\\
4 ~\mbox{bounded events} &\to 23917 ~\mbox{outcomes}\\
5 ~\mbox{bounded events} &\to 2244361 ~\mbox{outcomes}
\end{align*}
Clearly, simply superposing bounded events in this manner is not feasible, as 
it 
is unreasonable to expect that any given document would contain five or less 
events. In order to avoid generating such a large number of computed 
event-strings, it is necessary to add constraints where appropriate as to what 
strings may be considered allowable for a particular context.

\section{Constraints on Event-Strings}\label{constraints}
Two approaches to constraints may be implemented, which are not mutually 
exclusive. The first is to prevent unwanted strings from being generated, based 
on the nature of the operand strings, and the second is to remove disallowed 
strings from the set of outputs. The former approach is preferred from a 
computational standpoint, as there is less data to store and process.

We assume that every fluent we encounter has exactly one beginning and one 
ending -- that is, that events do not \textit{resume} once they have ended. 
Events of the same type may stop and start frequently, but by assuming that 
every instance of an event will have a uniquely identifying fluent, we can 
discard any strings which feature such a resumption.

Additionally, fluents 
may be referred to multiple times by different TLINKs in an annotated document, 
and we assume that they will be \textit{consistent} within the context of that 
document \textit{i.e.} if a relation holds between $e$ and $e'$, and a relation 
holds between $e'$ and $e''$, then both instances of $e'$ refer to the same 
fluent, and the two relations will not contradict each other.

These last two points are interesting in particular, as they lead to a specific 
kind of superposition between strings $s, s' \in \Sigma^+$ when some 
symbol $\alpha \in s$ is equal to some other symbol $\alpha' \in s'$. In this 
scenario, the symbols must unify when superposing the strings, in order to 
create a well-formed event-string in accordance with the above two constraints. 
To achieve this, when a symbol $\alpha$ in $s$ is also present in $s'$, and the 
asynchronous superposition of these strings is desired, padding is carried out 
as normal, but superposition is only permitted of those results of padding 
in which the indices of the matching symbols are equal:
\begin{align*}
\alpha_1\cdots\alpha_n \ \&\ 
\alpha'_1\cdots\alpha_n' & \ :=\
\{(\alpha_1\cup\alpha'_1)\cdots(\alpha_n\cup\alpha'_n) ~|~ \alpha_i = \alpha'_j 
\Rightarrow i = j\}
\end{align*}

Finally, we may also introduce simple binary constraints if external 
information is 
available (\textit{e.g.} ``the string may not begin with $e$"), and these might 
be 
simply intersected with the result of a superposition: $(s\ \&_{\bc} \ s') \cap 
C$, where $C$ represents the constraints to be applied.

\section{Grounding}
The TIMEBANK Corpus \citep{pustejovsky2003timebank} provides a large number of 
documents annotated with TimeML, from which we may extract the event data -- in 
particular the TLINKs. As mentioned in Section \ref{secER}, attempting to 
generate the possible worlds from just the bounded event instances alone is not 
possible, as there are just too many. Instead, we begin by looking at just 
those events which are linked to another by Allen Relation.

We translate these relations immediately to their corresponding event-strings, 
and superpose these according to the constraints mentioned in Section 
\ref{constraints}. In this way, we may generate a much smaller set of possible 
outcomes from a larger number of bounded events.

A drawback here is that for this to be effective, it relies on the events being 
heavily constrained by their interrelations. If there are too few TLINKs 
relative to the number of events, we still run into the problem of 
combinatorial explosion. An additional issue in computation of the 
superposition of events arises from unordered data leading to a much less 
efficient calculation of final results. For example, 
\fbox{\vph{$e'$}}\fbox{$a\vph{e'}$}\fbox{\vph{$e'$}}\fbox{$b\vph{e'}$}\fbox{\vph{$e'$}}
 $\&_{\bc}$ 
 \fbox{\vph{$e'$}}\fbox{$b\vph{e'}$}\fbox{\vph{$e'$}}\fbox{$c\vph{e'}$}\fbox{\vph{$e'$}}
 $\&_{\bc}$ 
 \fbox{\vph{$e'$}}\fbox{$c\vph{e'}$}\fbox{\vph{$e'$}}\fbox{$d\vph{e'}$}\fbox{\vph{$e'$}}
  and 
  \fbox{\vph{$e'$}}\fbox{$a\vph{e'}$}\fbox{\vph{$e'$}}\fbox{$b\vph{e'}$}\fbox{\vph{$e'$}}
  $\&_{\bc}$ 
  \fbox{\vph{$e'$}}\fbox{$c\vph{e'}$}\fbox{\vph{$e'$}}\fbox{$d\vph{e'}$}\fbox{\vph{$e'$}}
  $\&_{\bc}$ 
  \fbox{\vph{$e'$}}\fbox{$b\vph{e'}$}\fbox{\vph{$e'$}}\fbox{$c\vph{e'}$}\fbox{\vph{$e'$}}
  should produce the same, single output: 
 \fbox{\vph{$e'$}}\fbox{$a\vph{e'}$}\fbox{\vph{$e'$}}\fbox{$b\vph{e'}$}%
 \fbox{\vph{$e'$}}\fbox{$c\vph{e'}$}\fbox{\vph{$e'$}}\fbox{$d\vph{e'}$}\fbox{\vph{$e'$}},
  which they do. However, due to the respective orderings, the first sequence 
 will arrive at that conclusion much faster as 
 \fbox{\vph{$e'$}}\fbox{$a\vph{e'}$}\fbox{\vph{$e'$}}\fbox{$b\vph{e'}$}\fbox{\vph{$e'$}}
 $\&_{\bc}$ 
 \fbox{\vph{$e'$}}\fbox{$b\vph{e'}$}\fbox{\vph{$e'$}}\fbox{$c\vph{e'}$}\fbox{\vph{$e'$}}
  has one outcome 
  (\fbox{\vph{$e'$}}\fbox{$a\vph{e'}$}\fbox{\vph{$e'$}}\fbox{$b\vph{e'}$}\fbox{\vph{$e'$}}\fbox{$c\vph{e'}$}%
 \fbox{\vph{$e'$}}), which can immediately be asynchronously superposed with 
 \fbox{\vph{$e'$}}\fbox{$c\vph{e'}$}\fbox{\vph{$e'$}}\fbox{$d\vph{e'}$}\fbox{\vph{$e'$}}
  to produce the final output. However, 
  \fbox{\vph{$e'$}}\fbox{$a\vph{e'}$}\fbox{\vph{$e'$}}\fbox{$b\vph{e'}$}\fbox{\vph{$e'$}}
  $\&_{\bc}$ 
  \fbox{\vph{$e'$}}\fbox{$c\vph{e'}$}\fbox{\vph{$e'$}}\fbox{$d\vph{e'}$}\fbox{\vph{$e'$}}
   has 321 output strings, each of which must be individually asynchronously 
  superposed with 
  \fbox{\vph{$e'$}}\fbox{$b\vph{e'}$}\fbox{\vph{$e'$}}\fbox{$c\vph{e'}$}\fbox{\vph{$e'$}},
   only to come to the same conclusion, as only one of these results is 
  consistent.

One potential way to work around this pitfall is a grouping and ordering stage, 
where initially only events linked by some relation may be superposed, and only 
after the operand strings have been sorted to some optimal order. It may be 
prudent to only perform superposition at all on event strings which may be 
linked through one or more relations or shared fluents. In this way, new, 
underspecified events may be formed from the output strings. Consider the 
scenario with $e_1, \cdots, e_8 \in \digamma$, and the following Allen 
Relations:
\begin{align*}
e_1 <& ~e_4\\
e_1 ~\mathbf{m}& ~e_2\\
e_2 ~\mathbf{di}& ~e_3\\
e_5 ~\mathbf{s}& ~e_7\\
e_8 >& ~e_5
\end{align*}
Note that three obvious clusters of events can be seen: $e_1$, $e_2$, $e_3$, 
and $e_4$ 
are interlinked, as are $e_5$, $e_7$, and $e_8$, separately, while $e_6$ is not 
linked to either group. We might form the underspecified event groups $E_1$ and 
$E_2$ to refer to these first two clusters, at which point we may freely treat 
these groups as normal bounded events, and perform asynchronous superposition 
on their event strings, as well as with that of $e_6$ -- thus reducing the 
number of inputs from 8 to 3.

Additionally, various weightings might be considered as a method of 
priority-ordering in the case of a large $\digamma$, such as the number of 
component events in an underspecified event groups, or the number of relations 
linking to a particular event.
\section{Conclusion}
We have explored in this work the possibility of using strings as basis for 
modelling event data, motivated by their nature as computational entities. The 
operation of asynchronous superposition was described for composing strings 
which represent finite, bounded events, as well as its limits in terms of 
blowup when the operation is repeated in sequence. The problem is addressed by 
constraining the strings which may be superposed, with the thirteen unique 
Allen Relations forming the main part of these, as these can be found in 
annotated corpora such as TIMEBANK, using the TimeML standard.

Future work on this topic will examine using alternative model-bases to 
approach the same issue, such as using finite state automata, or a hybrid 
string/FSA approach, as well as the potential of employing parallel processing 
methods in order to tackle the combinatorial explosion that occurs in 
asynchronously superposing unconstrained bounded events.
%%%%%%%%%%%%%%%%%%%%%%%%%%%%%%%%%%%%%%%%%%%%%%%%%%%%%%%%%%%%%%%%%%%%%%%%%%%%%%%%
%                                                                              %
%                             Actual Document Ends                             %
%                                                                              %
%%%%%%%%%%%%%%%%%%%%%%%%%%%%%%%%%%%%%%%%%%%%%%%%%%%%%%%%%%%%%%%%%%%%%%%%%%%%%%%%
\bibliographystyle{chicago}{}
\bibliography{draft1}

\appendix
\section{Map-Reduce}
The Map-Reduce model is designed to process very large data sets efficiently, 
taking advantage of parallel processing methods and distributed computing. It 
uses a repeated multi-step algorithm which divides the data into chunks, 
converts these chunks into a number of key/value pairs using a \textit{Mapper} 
function, then distributes these pairs over a number of \textit{Reducers}. The 
Reducer will be called once for each unique key created by the Mapper, and will 
iterate through the values associated with it to produce zero or more outputs.

For example, a word-counting program may break a document into lines, each of 
which will be the input to a Mapper. The Mapper will break the line into words 
and output a key/value pair for each unique word, with the word itself as the 
key, and the number of its occurrences in the line as the value. The Reducer 
takes all the key/value pairs from all lines which have the same key and sums 
the values to output a total count of that word for the document.

In our asynchronous superposition of unconstrained bounded events, as we saw, a 
large amount of data is created in the form of the event-string outputs. This 
seems like a potential target for applying the Map-Reduce model.%, though the 
%fact that this paradigm should have a stateless mapper and reducer makes 
%things 
%a little difficult.

In theory, our Mapper might take as input the event-strings to be 
superposed and would return as key pairs of the strings, and as value the set 
of strings resulting from the superposition of the key-pair:
\begin{align*}
map(s_1 ~\&_{\bc}~ s_2 ~\&_{\bc}~ \cdots ~\&_{\bc}~ s_{n-1} ~\&_{\bc}~ s_n) = 
\{(s_1s_2, s_1 ~\&_{\bc}~ s_2), \cdots, (s_{n-1}s_n, s_{n-1} ~\&_{\bc}~ s_n)\}
\end{align*}
However, in this case, the Reducer has nothing to do, as each key is unique, 
and therefore pairs with matching keys cannot be collected and reduced to a 
single value. Similarly, if the key was chosen as just one of the string pair, 
rather than both.

Perhaps instead we take a particular string (\textit{e.g} the first) as a key, 
and then superposition of it with each other string as the values:
\begin{align*}
map(s_1 ~\&_{\bc}~ s_2 ~\&_{\bc}~ \cdots ~\&_{\bc}~ s_{n-1} ~\&_{\bc}~ s_n) = 
\{(s_1, s_1 ~\&_{\bc}~ s_2), \cdots, (s_1, s_1 ~\&_{\bc}~ s_n)\}
\end{align*}
We can use the Reducer now, but there's not a whole lot that the function can 
do, except collect the values. If it were to do that, it must be done in such a 
way that the result of each superposition from the previous step were kept 
separate, so we would have a collection of $n$ separate sets of generated 
strings, in order to prevent superposing two strings from the same earlier 
generation (which should be considered as parallel worlds). While this could 
potentially fulfill the goals of Map-Reduce, a large amount of redundant 
superposition has been performed, each time increasing the maximum length of 
the generated strings, making the computation less efficient at each 
stage.

As a third alternative, suppose the Mapper just used a generic key each time, 
and returned the result of the padding function.
\begin{align*}
map(s_1 ~\&_{\bc}~ s_2 ~\&_{\bc}~ \cdots ~\&_{\bc}~ s_{n-1} ~\&_{\bc}~ s_n) = 
\{(s_{key}, pad_k(s_1)), \cdots, (s_{key}, pad_k(s_n))\}
\end{align*}
where $s_{key}$ is a generic key, and $k = \mbox{length}(s_1) + \cdots + 
\mbox{length}(s_n) - (n - 1)$. In this scenario, the Reducer will be able to 
perform a simple, synchonous superposition on all the pairings of the values -- 
however, at just $n=5$, $k$ becomes $11$, and for a simple 
bounded event \fbox{\vph{$e'$}}\fbox{$e\vph{e'}$}\fbox{\vph{$e'$}} there are 
$66$ different strings returned by $pad_{11}$, meaning there may be $66^5$ 
different superposition operations to perform, which is over a billion.

Due to the fact that the Mapper and Reducer should both be pure, stateless 
functions, we are unfortunately quite limited in what applications it has in 
the context of asynchronous superposition. The best case would be to use a 
generic key for the Mapper, as in the last example above, but to have the 
values be superposed pairs of strings as in the first example. Then the Reducer 
could combine the generated strings in order to create new data and repeat the 
process. However, using a generic key in the manner described somewhat defeats 
the purpose of using the Mapper at all, and thus is not a good implementation 
of the Map-Reduce model either.
\end{document}