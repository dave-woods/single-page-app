\documentclass[a4paper,11pt,leqno]{article}

\usepackage{graphicx}  %%% for including graphics
\usepackage{url}       %%% for including URLs
\usepackage{times}
\usepackage{amssymb,amsmath,amscd}
\usepackage{natbib}
\usepackage[margin=25mm]{geometry}
\usepackage{fancyvrb}
\usepackage[hang,flushmargin]{footmisc}
\usepackage{makecell}

% Tim's custom commands
\newcommand{\bc}{{\rm b\!c}}
\newcommand{\unpad}{\mbox{{\rm unpad}}}
\newcommand{\vph}[1]{\vphantom{#1}}
\newcommand{\sta}[2]{\stackrel{#1}{#2}}
\newcommand{\sk}{{\sf k}}
\newcommand{\sv}{{\sf v}}
\newcommand{\rr}{{\rm r}}

% David's custom commands
\newcommand{\ebox}[1]{\fbox{$\vph{',}#1$}}
\newcommand{\eboxl}[1]{\fbox{$\vph{'}#1$}}
\newcommand{\eboxh}[1]{\fbox{$\vph{,}#1$}}
\newcommand{\eboxb}[1]{\fbox{$\vph{@}#1$}}

\title{Towards Efficient String Processing of Annotated Events}
\date{}

\author{David Woods\\
	ADAPT Centre\\
	Trinity College Dublin, Ireland\\
	\texttt{dwoods@tcd.ie}
	\and Tim Fernando, Carl Vogel\\
	Computational Linguistics Group \\
	Trinity Centre for Computing and Language Studies\\
	School of Computer Science and Statistics\\
	Trinity College Dublin, Ireland\\
	\texttt{tim.fernando@tcd.ie}, \texttt{vogel@tcd.ie}
}

\begin{document}
\maketitle
\thispagestyle{empty}
\pagestyle{empty}
%%%%%%%%%%%%%%%%%%%%%%%%%%%%%%%%%%%%%%%%%%%%%%%%%%%%%%%%%%%%%%%%%%%%%%%%%%%%%%%%
%                                                                              %
%                            Actual Document Begins                            %
%                                                                              %
%%%%%%%%%%%%%%%%%%%%%%%%%%%%%%%%%%%%%%%%%%%%%%%%%%%%%%%%%%%%%%%%%%%%%%%%%%%%%%%%
\section{Introduction}\label{intro}
This paper explores the use of strings as models to effectively represent event 
data such as might be found in a document annotated with TimeML. It is 
described how such data may be simply translated to strings, and how to infer 
information through operations on these strings. Strings are basic 
computational entities that can be more readily manipulated by machines than 
the infinite models of predicate logic. Finite sets of strings serve as finite 
models.

Given a finite set $A$ of fluents (predicates with an associated temporality), 
a string $s=\alpha_1 \cdots \alpha_n$ 
of subsets $\alpha_i$ of $A$ can be construed as a finite model 
consisting of $n$ moments of time $i \in \{1, \ldots, n\}$ with $\alpha_i$ 
specifying all fluents (in $A$) that (as unary predicates) hold at $i$.

Throughout this paper, a fluent $a \in A$ will be understood as naming 
an event, and the powerset $2^{A}$ of $A$ will serve as an alphabet $\Sigma = 
2^{A}$ of an 
\textit{event-string} $s \in \Sigma^+$. Such strings are finite models of 
Monadic Second Order logic, and are amenable to finite state methods. We will 
further restrict them in Section \ref{constraints}, with a focus on using Allen 
Relations adopted in TimeML, in order to analyse inference over a finite search 
space.

An event-string $\alpha_1 \cdots \alpha_n$ is read from left to right 
chronologically, 
so that any predicates which hold at the moment at index $i$ are understood to 
have held before another moment indexed by $j$ if and only if $i < j$. The 
precise duration of each moment is taken as unimportant in the current 
discussion, 
and thus the strings model an inertial world, whereby \textit{change} is the 
only mark of progression from one moment to the next -- ``But neither does time 
exist without change" (Aristotle, \textit{Physics IV}). Thus, if $\alpha_i = 
\alpha_{i+1}$ for any 
$1 \leq i < n$, then either $\alpha_i$ or $\alpha_{i+1}$ may be safely deleted 
from $s$ without 
affecting the interpretation of the string, as the remaining symbol is simply 
taken as representing a longer moment. This operation of removing 
repetition from 
the event-string is known as \textit{block compression} 
\citep{fernando2016prior}. The inverse of this process introduces repeated 
elements in an event-string for greater flexibility in manipulating strings. 
These operations are detailed in Section \ref{sp and bc}.

We see that strings may provide useful finite models for event data, once 
sufficiently constrained. This is in order to avoid a large combinatorial 
blowup when reconciling information from different strings.

\section{Superposition and Block Compression}\label{sp and bc}
With two strings $s$ and $s'$ of the same length $n$
built from an alphabet $\Sigma$, the powerset of some fixed set 
$A$,
the \textit{superposition} $s ~\&~ s'$ of $s$ and $s'$ is their componentwise 
union:
\begin{align}
\alpha_1\cdots\alpha_n \ \&\ 
\alpha'_1\cdots\alpha_n' & \ :=\
(\alpha_1\cup\alpha'_1)\cdots(\alpha_n\cup\alpha'_n)
\end{align}
For convenience of notation, we will use boxes rather than curly braces 
$\{$~$\}$ to represent sets in $\Sigma$, such that each symbol $\alpha$ in a 
string $s$ corresponds to exactly one box. For example, with $a, b, c, d \in 
A$:
\begin{align}
\ebox{a}\ebox{c} \ \& \ \ebox{b}\ebox{d} \ = \
\ebox{a, b}\ebox{c, d} \ \in\ \Sigma^2
\end{align}
Extending superposition to languages $L$ and $L'$ over the same alphabet is a 
simple matter of 
collecting the superpositions of strings of equal length from each language: 
\begin{align}
L ~\&~ L' & \ :=\ \bigcup_{n\geq 0}
\{s~\&~s'\ | \ s\in L\cap \Sigma^n\mbox{ and }s'\in L'\cap \Sigma^n\}
\end{align}
For example, $L ~\&~ \eboxb{}^\ast = L$.
If $L$ and $L'$ are regular languages computed by finite automata
with transitions $\to$ and $\to'$, then the superposition $L~\&~ L'$ is
a regular language computed by a finite automaton with transitions
$\Rightarrow$
formed by running $\to$ and $\to'$ in lockstep
according to the rule 
\begin{align}
\frac{q \sta{\alpha}{\to} r   \hspace{.4in} q' \sta{\alpha'}{\to'} r'}{
	(q,q') \sta{\alpha\cup\alpha'}{\Rightarrow} (r,r')}
\end{align}
A disadvantage of this operation is that it requires the string operands to be 
of equal length, which is an overly specific case. In order to generalise this 
procedure to strings of arbitrary lengths, we may 
manipulate the strings to move away from the synchrony of the lockstep 
procedure. One such manipulation is that 
we can cause a string $s=\alpha_1\cdots\alpha_n$ to \textit{stutter} such that 
$\alpha_i=\alpha_i+1$ for some integer $0 < i < n$. For example, 
\eboxl{a}\eboxl{a}\eboxl{a}\eboxl{c}\eboxl{c}
 is a stuttering version of 
\eboxl{a}\eboxl{c}. If 
a string does not 
stutter, it is \textit{stutterless}, and we can transform a stuttering string 
to this 
state by using ``block compression":
\begin{align}
\bc(s)  &\ \ :=\ \
\left\{ \begin{array}{ll}
s & \mbox{ if }~length(s)\leq 1 \\
\bc({\alpha}s')  & \mbox{ if } s={\alpha}{\alpha}s'\\
\alpha ~{\bc}({\alpha'}{s'})  
& \mbox{ if } s=\alpha{\alpha'}{s'} \mbox{ with } \alpha\neq\alpha'
\end{array}
\right. 
\end{align}
This function can be applied multiple times to a string, but the output will 
not change after the first application: $\bc(\bc(s)) = \bc(s)$. We can also use 
the inverse of this function to generate infinitely many stuttering strings:
\begin{align}
\bc^{-1}(\ebox{a}\ebox{c}) = 
\{ \ebox{a}\ebox{c},
\ebox{a}\ebox{a}\ebox{c},
\ebox{a}\ebox{c}\ebox{c}, 
\ebox{a}\ebox{a}\ebox{c}\ebox{c}, 
\ldots \}
\end{align}
We can say that any of the strings generated by this inverse block compression 
are \textit{$\bc{}$-equivalent}. Precisely, a string $s'$ is $\bc{}$-equivalent 
to a 
string 
$s$ iff $s' \in \bc^{-1}\bc(s)$.

We can now define the \textit{asynchronous superposition} $s~\&_*~s'$ of 
strings $s$ and 
$s'$ as the (provably) \textit{finite} set obtained by block compressing the 
\textit{infinite} 
language generated by superposing the strings which are $\bc$-equivalent to $s$ 
and $s'$:
\begin{align}
s\ \&_* \ s' & \ := \
\{\bc(s'') \ | \ s''\in \bc^{-1}\bc(s)\ \& \ \bc^{-1}\bc(s')\}
\end{align}
For example, \ebox{a}\ebox{c} $\&_*$ 
\ebox{b}\ebox{d} will comprise 
three strings:
\begin{align}
\{ \ebox{a, b}\ebox{c, d},
\ebox{a, b}\ebox{a, d}\ebox{c, d},
\ebox{a, b}\ebox{b, c}\ebox{c, d} \}
\end{align}
In order to avoid generating all possible strings when using the inverse block 
compression, we introduce an upper bound to the length of the strings which 
will be superposed. It can be shown that with two strings of length $n$ and 
$n'$, the longest $\bc$-unique string (one which has no shorter 
$\bc$-equivalent strings) produced through asynchronous superposition will be 
of length $n+n'-1$.

\section{Upper Bound on Asynchronous Superposition}\label{upper}
For all $s,s'\in \Sigma^{\ast}$, we define a finite set
$s~\hat{\&}~s'$ of strings over $\Sigma$ with enough of 
the strings in $\bc^{-1}\bc(s)~\&~\bc^{-1}\bc(s')$
to form $s~\&_*~s'$.
The definition proceeds by induction on $s$ and $s'$, with
\begin{subequations}
\begin{align}
\epsilon\ \hat{\&}\ \epsilon \ :=& \ \{\epsilon\}\\
\epsilon\ \hat{\&}\ s \ :=& \ \emptyset\ \ \mbox{ for } s\neq\epsilon\\
s\ \hat{\&}\ \epsilon \ :=& \ \emptyset\ \ \mbox{ for } s\neq\epsilon
\end{align}
\end{subequations}
and for all $\alpha,\alpha'\in \Sigma$,
\begin{align}
\alpha s~\hat{\&}~\alpha's' \ :=&~\{(\alpha\cup\alpha')s''\ | \ s''\in (\alpha 
s~\hat{\&}~s') \cup 
(s~\hat{\&}~\alpha's') \cup (s~\hat{\&}~s')\}
\end{align}
Note that a string in $s\ \hat{\&}\ s'$ might stutter, even if neither of the 
operands $s$ or $s'$ do
(\textit{e.g.} $\ebox{a,c}\ebox{a,c}\in
\ebox{a}\ebox{c}\ \hat{\&}\ 
\ebox{c}\ebox{a}$)
-- however, it can be made stutterless through block compression.
\paragraph{Proposition 1.} {\sl For all  $s, s' \in \Sigma^+$
	and all $s''\in s\ \hat{\&}\ s'$,}
\begin{align}
\mbox{length}(s'')\ \leq\ \mbox{length}(s) + \mbox{length}(s') -1
\end{align}
\paragraph{Proposition 2.} {\sl For all $s,s'\in \Sigma^+$,}
\begin{align}
s\ \hat{\&}\ s' & \ \subset\ \bc^{-1}\bc(s)\ \&\ \bc^{-1}\bc(s')
\end{align}
and
\begin{align}
\{\bc(s'')\ | \ s''\in s\ \hat{\&}\ s'\}
& \ = \   s\ \&_*\ s'
\end{align}
Now, for any integer $k > 0$ and string $s = \alpha_1\cdots\alpha_n$ over 
$\Sigma$, we introduce a new 
function 
$pad_k$ which will generate the set of strings with length $k$ which are 
$\bc$-equivalent to $s$:
\begin{subequations}
\begin{align}
pad_k(\alpha_1\cdots\alpha_n)~:=&~~\alpha_1^+\cdots\alpha_n^+\ \cap\ \Sigma^k \\
=&~~\{
\alpha_1^{k_1}\cdots\alpha_n^{k_n}\ | \
k_1,\ldots,k_n\geq 1
\mbox{ and } \sum_{i=1}^n k_i = k \}\\
\subset&~~\bc^{-1}\bc(\alpha_1\cdots\alpha_n)
\end{align}
\end{subequations}
For example, $pad_{4}(\eboxl{a}\eboxl{c})$ will 
generate \{
\eboxl{a}\eboxl{a}\eboxl{a}\eboxl{c}, 
\eboxl{a}\eboxl{a}\eboxl{c}\eboxl{c}, 
\eboxl{a}\eboxl{c}\eboxl{c}\eboxl{c} \}. We 
can use this new function in our 
calculation of asynchronous superposition, to limit the generation of strings 
from the inverse block compression step. Since we know from Proposition 1 that 
the maximum possible 
length we might need is $n+n'-1$, we can use this value in the $pad$ function 
to just 
generate the strings of that length, giving us a new definition of asynchronous 
superposition:
\paragraph{Corollary 3.} {\sl For any $s,s'\in \Sigma^+$
	with nonzero lengths $n$ and $n'$ respectively,}
\begin{align}
s \ \&_* \ s' & \ = \
\{\bc(s'')\ | \ s''\in pad_{n+n'-1}(s)\ \& \ pad_{n+n'-1}(s')\}
\end{align}
Neither $s\ \hat{\&}\ s'$ nor $pad_{n+n'-1}(s)\ \& \ pad_{n+n'-1}(s')$
need be a subset of the other, even though,
under the assumptions of Corollary 3, 
both sets block compress to $s \ \&_* \ s'$.

\section{Event Representation}\label{event rep}
Now we may use asynchronous superposition to generate the 13 strings in 
\eboxl{}\eboxl{e}\eboxl{} 
$\&_*$ 
\eboxl{}\eboxl{e'}\eboxl{}, each of which corresponds to one 
of the unique interval relations in \cite{allen1983maintaining}. No more than 
one of these relations may hold between any two fluents, and thus each of the 
13 generated event-strings exists in a distinct possible ``world".
We use the empty box \eboxl{} as a string of length 1 (not to be confused 
with the empty string $\epsilon$, which is length 0) to bound events, allowing 
us to represent the fact that they are finite -- they have a beginning and 
ending point. It is prudent to assume that we will deal only with finite event 
data, such that there are no fluents which do not have both an associated 
start-point and end-point. If such a non-finite fluent without a begining and 
ending were to occur, it would could trivially appear in every position in the 
event-string. 

The bounding boxes represent the time 
before and after the event occurs, during which no other fluents $a \in A$ 
are mentioned. The 
event-strings associated with the Allen Relations are laid out 
below:
\begin{center}
	\begin{tabular}{ l@{\hskip 1in}c@{\hskip 1in}r }
		$e~\mathbf{=}~e'$ & \ebox{}\ebox{e, e'}\ebox{} & equal\\[0.6em]		
		$e~\mathbf{s}~e'$ & \ebox{}\ebox{e, e'}\ebox{e'}\ebox{} & 
		starts\\[0.6em]
		$e~\mathbf{si}~e'$ & \ebox{}\ebox{e, e'}\ebox{e}\ebox{} & 		
		starts~(inverse)\\[0.6em]
		$e~\mathbf{f}~e'$ & \ebox{}\ebox{e'}\ebox{e, e'}\ebox{} & 		
		finishes\\[0.6em]
		$e~\mathbf{fi}~e'$ & \ebox{}\ebox{e}\ebox{e, e'}\ebox{} &		
		finishes~(inverse)\\[0.6em]
		$e~\mathbf{d}~e'$ & \ebox{}\ebox{ e'}\ebox{e, e'}\ebox{ e'}\ebox{} & 
		during\\[0.6em]
		$e~\mathbf{di}~e'$ & \ebox{}\ebox{ e}\ebox{e, e'}\ebox{e}\ebox{} & 
		during~(inverse)\\[0.6em]
		$e~\mathbf{o}~e'$ & \ebox{}\ebox{ e}\ebox{e, e'}\ebox{ e'}\ebox{} & 
		overlaps\\[0.6em]
		$e~\mathbf{oi}~e'$ & \ebox{}\ebox{e'}\ebox{e, e'}\ebox{e}\ebox{} & 
		overlaps~(inverse)\\[0.6em]
		$e~\mathbf{m}~e'$ & \ebox{}\ebox{ e}\ebox{e'}\ebox{} & meets\\[0.6em]
		$e~\mathbf{mi}~e'$ & \ebox{}\ebox{e'}\ebox{ e}\ebox{} & 
		meets~(inverse)\\[0.6em]
		$e~\mathbf{<}~e'$ & \ebox{}\ebox{e}\ebox{}\ebox{ e'}\ebox{} & 
		before\\[0.6em]
		$e~\mathbf{>}~e'$ & \ebox{}\ebox{e'}\ebox{}\ebox{ e}\ebox{} & 
		after
	\end{tabular}
\end{center}
These Allen Relations are included in the attributes of ISO-TimeML 
\citep{pustejovsky2010iso}, the standard 
markup language used for event annotation in texts, as TLINKs. By extracting 
the TLINKs from an annotated document, and translating them to our event-string 
representation (see Section \ref{application}), we may begin to reason about 
the relationships between 
annotated events which do not have an associated TLINK in the markup. For 
example, the document may give us a relation between events $e$ and $e'$, and 
another relation between $e'$ and $e''$, and from this we may infer the 
possible relations between $e$ and $e''$.

As asynchronous superposition is commutative and associative, we may superpose 
arbitrary numbers of event-strings: $s_1 ~\&_*~ \cdots ~\&_*~ s_n$. We can show 
that superposing $n$ unconstrained bounded event-strings will generate strings 
of maximum length $2n + 1$.\footnote{
\begin{math}
%\mbox{Proof by induction:}\\
\mbox{Let each string to be superposed } s_i \in \{ s_1, \ldots, s_n 
\}~\mbox{be}~
\ebox{}\ebox{e_i}\ebox{}\mbox{, with each }e_i \in A\\
\mbox{For } n = 2\mbox{: }s_1~\&_*~s_2\\
\mbox{From Proposition 1, the maximum length 
of the result is } 3 + 3 - 1 = 5 = 2(2) + 1\\
\mbox{We assume true for }n = p\mbox{, thus the maximum length of }s_1 ~\&_*~ 
\cdots ~\&_*~ s_p\mbox{ is }2(p) + 1\\
\mbox{Next, we prove for }n = p + 1\mbox{:}\\
s_1 ~\&_*~ \cdots ~\&_*~ s_{p+1} = s_1 ~\&_*~ \cdots ~\&_*~ s_p~\&_*~ 
s_{p+1} = s_{1 \ldots p}~\&_*~s_{p+1}\\
\mbox{From Proposition 1, the maximum length of the result is }(2(p) + 1) + 3 
- 1 = 2(p + 1) + 1\\
\mbox{Thus true for }p+1\mbox{, and by induction, true for any }n \geq 2
\end{math}}
Note, however, that superposing even a relatively small number of 
unconstrained bounded events leads to a massive combinatorial blowup in the 
number of outcomes, or possible worlds, as each event-string generated from one 
superposition (\textit{e.g.} $s_1 ~\&_*~ s_2$) will in turn be superposed with 
each generated from another (\textit{e.g.} $s_3 ~\&_*~ s_4$). Additionally, 
with each fluent, the maximum possible length of the strings grows, meaning a 
larger set of strings will be generated at the $pad$ stage. The combinatorics 
for $n$ unconstrained bounded events are as follows, up to $n = 5$:
\begin{align*}
2 ~\mbox{bounded events} &\to 13 ~\mbox{outcomes}\\
3 ~\mbox{bounded events} &\to 409 ~\mbox{outcomes}\\
4 ~\mbox{bounded events} &\to 23917 ~\mbox{outcomes}\\
5 ~\mbox{bounded events} &\to 2244361 ~\mbox{outcomes}
\end{align*}
Clearly, simply superposing bounded events in this manner is not feasible, as 
it is unreasonable to expect that any given document should contain five or 
fewer events. In order to avoid generating such a large number of computed 
event-strings, it is necessary to add constraints where appropriate to limit the
strings that may be considered allowable for a particular context.

Interestingly, because each unconstrained bounded event-string 
\eboxb{}\eboxb{e}\eboxb{} contains exactly one fluent, we may determine the 
maximum possible length of a string generated by superposition, $2n + 1$, from 
the size of the set $A$ of fluents, where $n = |A|$. By keeping track of $|A|$, 
we ensure that the length of the string will always be finite, opening up the 
possibility of using methods from constraint satisfaction, exploiting the 
finite search space.

\section{Constraints on Event-Strings}\label{constraints}
Two approaches to constraints may be implemented, which are not mutually 
exclusive. The first is to prevent unwanted strings from being generated, based 
on the nature of the operand strings, and the second is to remove disallowed 
strings from the set of outputs. The former approach is preferred from a 
computational standpoint, as there is less data to store and process. For 
either, we define some properties of what we may consider to be a 
\textit{well-formed event-string}.

We assume that every fluent we encounter has exactly one beginning and one 
ending -- that is, that events do not \textit{resume} once they have ended. 
Events of the same type may stop and start frequently, but by assuming that 
every instance of an event will have a uniquely identifying fluent, we can 
discard any strings which feature such a resumption.\footnote{We adopt 
simplifying assumptions made in Allen Relations, though it should be noted that 
the distinction between event instances event types is not imposed by the 
event-string framework itself, allowing for discontinuous events (such as 
\textit{judder}) in future work} In this way, fluents are 
\textit{interval-like}. We define the function 
$\rho_{X}$ on strings of sets to component-wise intersect with $X$ for 
any $X 
\subseteq A$ \citep{fernando2016prior}:
\begin{align}
\rho_{X}(\alpha_1 \cdots \alpha_n) := (\alpha_1 \cap X) \cdots 
(\alpha_n 
\cap 
X)
\end{align}
Applying block compression to an event-string which has been reduced with 
$\rho_{\{a\}}$ should produce a single string: \eboxb{}\eboxb{a}\eboxb{}. For 
example, with $a, b \in A$:
\begin{align}
\bc{}(\rho_{\{a\}}(\ebox{}\ebox{a}\ebox{a, b}\ebox{b}\ebox{})) = 
\ebox{}\ebox{a}\ebox{}
\end{align}
Additionally, fluents may be referred to multiple times by different TLINKs in 
an annotated document, and we assume that they will be \textit{consistent} 
within the context of that document \textit{i.e.} if a relation holds between 
$e$ and $e'$, and a relation holds between $e'$ and $e''$, then both instances 
of $e'$ refer to the same fluent. In this case, if a relation also holds 
between $e$ and $e''$, then this relation should not contradict the other two 
relations. For example, if $e~>~e'$ and $e'~>~e''$, then it should be 
impossible for a well-formed event-string to also have the relation $e~<~e''$, 
as this would break the interval-like fluent constraint mentioned above.

These last two points are interesting in particular, as they lead to a specific 
kind of superposition between strings $s, s' \in \Sigma^+$ when some 
symbol $\alpha \in s$ is equal to some other symbol $\alpha' \in s'$. In this 
scenario, the symbols must unify when superposing the strings, in order to 
create a well-formed event-string in accordance with the above two constraints. 
To achieve this, when a symbol $\alpha$ in $s$ is also present in $s'$, and the 
asynchronous superposition of these strings is desired, padding is carried out 
as normal, but superposition is only permitted of those results of padding 
in which the indices of the matching symbols are equal. To do otherwise would 
permit event-strings which are not well-formed.

Finally, we may also introduce further constraints if external 
information is 
available, and these might be simply intersected with the result of a 
superposition: $(s\ \&_* \ s') \cap C$, where $C$ represents the constraints to 
be applied, for example, ``$e$ is among the first events to occur in the string 
$s$" (true iff $s ~\&~ $\eboxb{}\eboxb{e}\eboxb{}$^\ast = s$). This allows for 
extension 
beyond Allen Relations in the 
future.

\section{Application to TimeML}\label{application}
The TIMEBANK Corpus \citep{pustejovsky2003timebank} provides a large number of 
documents annotated using TimeML, from which we may extract the event data -- 
in particular the TLINKs. These elements in the markup indicate the relation 
holding between two fluents which was found in the text of the document. Though 
not every fluent will necessarily be linked with another in this manner, a 
majority will be.  As mentioned in Section \ref{event rep}, attempting to 
generate all of the possible worlds becomes difficult when using just the 
unconstrained bounded event-strings alone, as there are just too many (rarely, 
if ever, five or fewer). Instead, we begin by looking at just those fluents 
which are linked to another by Allen Relation.

As each relation corresponds exactly to one possible model, we translate them 
immediately to the appropriate event-strings, and superpose these according to 
the constraints mentioned in Section \ref{constraints}. This allows us to avoid 
simply superposing based on the fluents, and bypasses having to generate the 
initial 13 possibilities. In this way, we may generate a much smaller set of 
possible outcomes from a larger number of bounded events.

According to the TimeML specification \citep{timeml2005timeml}, a TLINK is 
required to have the following attributes: either a \texttt{timeID} or 
\texttt{eventInstanceID} attribute, referring to some fluent in the text, as 
well as either a \texttt{relatedToTime} or \texttt{relatedToEventInstance} 
attribute, which will refer to another fluent, and also a \texttt{relType} 
attribute, declaring the relation between the two fluents. Other attributes are 
optional and not relevant to the current discussion.

In order to give a more concrete understanding, let us take a pair of examples 
of TLINKs from a TimeML document: 

\begin{align}
\mbox{\texttt{<TLINK lid="l9" relType="IS\_INCLUDED" timeID="t1"}}\\ 
\label{eq:timeml1}
\mbox{\texttt{relatedToEventInstance="ei9" origin="USER"/>}} \notag\\
\mbox{\texttt{<TLINK lid="l10" relType="BEFORE" 
eventInstanceID="ei9"}}\\\label{eq:timeml2}
\mbox{\texttt{relatedToEventInstance="ei10" origin="USER"/>}} \notag
\end{align}

\noindent
The Allen Relation specified by \texttt{relType} will correspond to 
one of the 13 event-strings given in Section \ref{event rep}, which will 
in 
turn determine which formula the new event-string will satisfy. Example 
\ref{eq:timeml1} above gives the event-string \ebox{}\ebox{ei9}\ebox{t1, 
ei9}\ebox{t1}\ebox{}, and example \ref{eq:timeml2} gives 
\ebox{}\ebox{ei9}\ebox{}\ebox{ei10}\ebox{}.\footnote{Using 
	https://www.scss.tcd.ie/$\sim$dwoods/timeml/ to quickly extract TLINKs from 
	a 
	TimeML document and translate them to event-strings -- a non-trivial 
	extension of this program which computes results of superposition is 
	possible}

A drawback here is that for this to be effective, it relies on the events being 
heavily constrained by their interrelations. If there are too few TLINKs 
relative to the number of events, we still run into the problem of 
combinatorial explosion.

An additional issue in computation of the 
superposition of events arises from unordered data leading to a much less 
efficient calculation of final results. For example, 
\eboxl{}\eboxl{a}\eboxl{}\eboxl{b}\eboxl{}
 $\&_*$ 
 \eboxl{}\eboxl{b}\eboxl{}\eboxl{c}\eboxl{}
 $\&_*$ 
 \eboxl{}\eboxl{c}\eboxl{}\eboxl{d}\eboxl{}
  and 
  \eboxl{}\eboxl{a}\eboxl{}\eboxl{b}\eboxl{}
  $\&_*$ 
  \eboxl{}\eboxl{c}\eboxl{}\eboxl{d}\eboxl{}
  $\&_*$ 
  \eboxl{}\eboxl{b}\eboxl{}\eboxl{c}\eboxl{}
  should produce the same, single output: 
 \eboxl{}\eboxl{a}\eboxl{}\eboxl{b}%
 \eboxl{}\eboxl{c}\eboxl{}\eboxl{d}\eboxl{},
  which they do. However, due to the respective orderings, the first sequence 
 will arrive at that conclusion much faster as 
 \eboxl{}\eboxl{a}\eboxl{}\eboxl{b}\eboxl{}
 $\&_*$ 
 \eboxl{}\eboxl{b}\eboxl{}\eboxl{c}\eboxl{}
  has one outcome 
  (\eboxl{}\eboxl{a}\eboxl{}\eboxl{b}\eboxl{}\eboxl{c}%
 \eboxl{}), which can immediately be asynchronously superposed with 
 \eboxl{}\eboxl{c}\eboxl{}\eboxl{d}\eboxl{}
  to produce the final output. However, 
  \eboxl{}\eboxl{a}\eboxl{}\eboxl{b}\eboxl{}
  $\&_*$ 
  \eboxl{}\eboxl{c}\eboxl{}\eboxl{d}\eboxl{}
   has 321 output strings, each of which must be individually asynchronously 
  superposed with 
  \eboxl{}\eboxl{b}\eboxl{}\eboxl{c}\eboxl{},
   only to come to the same conclusion, as only one of these results is 
  consistent.

One potential way to work around this pitfall is a grouping and ordering stage, 
where initially only events linked by some relation may be superposed, and only 
after the operand strings have been sorted to some optimal order, whereby the 
event-strings with the most shared fluents are grouped. It may be 
prudent to only perform superposition at all on event-strings which may be 
linked through one or more relations or shared fluents. In this way, new, 
underspecified events may be formed from the output strings. Consider the 
scenario with $e_1, \ldots, e_8 \in A$, and the following Allen 
Relations:
\begin{align*}
e_1 <& ~e_4\\
e_1 ~\mathbf{m}& ~e_2\\
e_2 ~\mathbf{di}& ~e_3\\
e_5 ~\mathbf{s}& ~e_7\\
e_8 >& ~e_5
\end{align*} 
Let us cluster the fluents as follows: for each fluent $a \in A$, fix a set $P 
= \{ a \}$, and a set $S$ whose members are these sets $P$.
\begin{align}
S = \{\{e_1\},\{e_2\},\{e_3\},\{e_4\},\{e_5\},\{e_6\},\{e_7\},\{e_8\}\}
\end{align}
Next, check for an Allen Relation between each pair of fluents $a$ and $a'$ -- 
if 
a relation exists, add the fluent $a'$ to the set $P$.
\begin{align}
S' = \{\{e_1, e_4, e_2\},\{e_2, e_3\}, \{e_3\}, \{e_4\}, \{e_5, e_7\}, \{e_6\}, 
\{e_7\}, \{e_8, e_5\}\}
\end{align}
Finally, for each pair of sets $P, Q$, if $|P ~\cap~ Q| > 0$, form $R = P 
~\cup~Q$, adding $R$ to the set $S''$ and discarding $P, Q$. Add the remaining 
sets 
from $S'$ to $S''$.
\begin{align}
S'' = \{\{e_1, e_4, e_2, e_3\}, \{e_5, e_7, e_8\}, \{e_6\}\}
\end{align}
We might form the underspecified event groups $E_1$ and 
$E_2$ to refer to these first two clusters, at which point we may freely treat 
these groups as normal bounded events, and perform asynchronous superposition 
on their event-strings, as well as with that of $e_6$ -- reducing the 
number of inputs from 8 to 3.

Additionally, various weightings might be considered as a method of 
priority-ordering in the case of a large $A$, such as the number of 
component events in an underspecified event group, or the number of relations 
linking to a particular event.
\section{Conclusion}\label{conclusion}
We have explored in this work the possibility of using strings as basis for 
modelling event data, motivated by their nature as computational entities. The 
operation of asynchronous superposition was described for composing strings 
which represent finite, bounded events, as well as its limits in terms of 
blowup when the operation is repeated in sequence. The problem is addressed by 
constraining the strings which may be superposed, with the 13 unique 
Allen Relations forming the main part of these, as these can be found in 
annotated corpora such as TIMEBANK, using the TimeML standard.

Future work on this topic will examine using alternative model-bases to 
approach the same issue, such as using finite state automata, or a hybrid 
string/FSA approach, as well as the potential of employing methods from 
distributed computing in order to tackle the combinatorial explosion that 
occurs in asynchronously superposing unconstrained bounded events.
%%%%%%%%%%%%%%%%%%%%%%%%%%%%%%%%%%%%%%%%%%%%%%%%%%%%%%%%%%%%%%%%%%%%%%%%%%%%%%%%
%                                                                              %
%                             Actual Document Ends                             %
%                                                                              %
%%%%%%%%%%%%%%%%%%%%%%%%%%%%%%%%%%%%%%%%%%%%%%%%%%%%%%%%%%%%%%%%%%%%%%%%%%%%%%%%
\bibliographystyle{chicago}{}
\bibliography{draft1}
\appendix
\section{Transitivity Table}
\cite{allen1983maintaining} gives a transitivity table showing the inferred 
possible relations between two events $a$ and $c$, given the relation between 
each and an intermediary event, $b$. Each cell of the table shows simply the 
symbol which represent the binary relation -- we may improve on the readability 
of this by showing explicitly the event-string(s) formed by the asynchronous 
superposition in each case. A fragment of the entire table is shown below:

\setlength{\tabcolsep}{3pt}
\renewcommand{\arraystretch}{2}
\vspace{1em}
\noindent
\begin{tabular}[h]{| c | c | c | c | c |}
	\hline
	  & \thead{``before"\\\ebox{}\ebox{b}\ebox{}\ebox{c}\ebox{}} & 
	  \thead{``during"\\\ebox{}\ebox{c}\ebox{b,c}\ebox{c}\ebox{}} & 
	  \thead{``overlaps"\\\ebox{}\ebox{b}\ebox{b,c}\ebox{c}\ebox{}} & 
	  \thead{``starts"\\\ebox{}\ebox{b,c}\ebox{c}\ebox{}}\\
	\hline
	\thead{``before"\\\ebox{}\ebox{a}\ebox{}\ebox{b}\ebox{}} & 
	\thead{\ebox{}\ebox{a}\ebox{}\ebox{b}\ebox{}\ebox{c}\ebox{}} & 
	\thead{\ebox{}\ebox{a}\ebox{}\ebox{c}\ebox{b,c}\ebox{c}\ebox{},\\ 
		\ebox{}\ebox{a}\ebox{a,c}\ebox{c}\ebox{b,c}\ebox{c}\ebox{},\\ 
		\ebox{}\ebox{a}\ebox{c}\ebox{b,c}\ebox{c}\ebox{},\\ 
		\ebox{}\ebox{c}\ebox{a,c}\ebox{c}\ebox{b,c}\ebox{c}\ebox{},\\ 
		\ebox{}\ebox{a,c}\ebox{c}\ebox{b,c}\ebox{c}\ebox{}} & 
	\thead{\ebox{}\ebox{a}\ebox{}\ebox{b}\ebox{b,c}\ebox{c}\ebox{}} & 
	\thead{\ebox{}\ebox{a}\ebox{}\ebox{b,c}\ebox{c}\ebox{}}\\
	\hline
	\thead{``during"\\\ebox{}\ebox{b}\ebox{a,b}\ebox{b}\ebox{}} & 
	\thead{\ebox{}\ebox{b}\ebox{a,b}\ebox{b}\ebox{}\ebox{c}\ebox{}} & 
	\thead{\ebox{}\ebox{c}\ebox{b,c}\ebox{a,b,c}\ebox{b,c}\ebox{c}\ebox{}} & 
	\thead{\ebox{}\ebox{b}\ebox{a,b}\ebox{b}\ebox{b,c}\ebox{c}\ebox{},\\
		\ebox{}\ebox{b}\ebox{a,b}\ebox{a,b,c}\ebox{b,c}\ebox{c}\ebox{},\\
		\ebox{}\ebox{b}\ebox{a,b}\ebox{b,c}\ebox{c}\ebox{},\\
		\ebox{}\ebox{b}\ebox{b,c}\ebox{a,b,c}\ebox{b,c}\ebox{c}\ebox{},\\
		\ebox{}\ebox{b}\ebox{a,b,c}\ebox{b,c}\ebox{c}\ebox{}} & 
	\thead{\ebox{}\ebox{b,c}\ebox{a,b,c}\ebox{b,c}\ebox{c}\ebox{}}\\
	\hline
	\thead{``overlaps"\\\ebox{}\ebox{a}\ebox{a,b}\ebox{b}\ebox{}} & & & & \\
	\hline
	\thead{``starts"\\\ebox{}\ebox{a,b}\ebox{b}\ebox{}} & & & & \\
	\hline
\end{tabular}

\vspace{1em}
\noindent
Here and in the original table, only three events are mentioned: $a$, $b$, and 
$c$. We can see that the asynchronous superposition of a string $s_{a,b}$ 
mentioning $a$ and $b$ with a string $s_{b,c}$ mentioning $b$ and $c$ gives a 
language $L$ of strings mentioning all three events. Applying the reduct 
$\rho_{\{a,b\}}$ to any string in $L$ should give back exactly $s_{a,b}$, and 
likewise applying $\rho_{\{b,c\}}$ to any string in $L$ should give back 
exactly $s_{b,c}$. It should, in theory, be possible to generalise this to any 
number of events.
\end{document}