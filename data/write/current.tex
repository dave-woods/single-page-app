\documentclass[a4paper,11pt]{article}

\usepackage{graphicx}  %%% for including graphics
\usepackage{url}       %%% for including URLs
\usepackage{times}
\usepackage{amssymb,amsmath,amscd}
\usepackage{natbib}
\usepackage[margin=25mm]{geometry}

% Tim's custom commands
\newcommand{\bc}{{\rm b\!c}}
\newcommand{\unpad}{\mbox{{\rm unpad}}}
\newcommand{\vph}[1]{\vphantom{#1}}
\newcommand{\sta}[2]{\stackrel{#1}{#2}}
\newcommand{\sk}{{\sf k}}
\newcommand{\sv}{{\sf v}}
\newcommand{\rr}{{\rm r}}

\title{Title}
\date{}

\author{Example Author\\
	Affiliation\\
	\texttt{example@email.org}
	\and Someone Else\\
	Another Affiliation\\
	\texttt{another@email.org}
}

\begin{document}
\maketitle
\thispagestyle{empty}
\pagestyle{empty}

%\section{Structure}
%
%\begin{itemize}
%	\item Why strings are useful $\to$ bring information together
%	\item Can't presuppose translation of Allen Relations to strings
%	\item Block compression
%	\item Superposition
%	\item Exploring usefulness of having longer strings for greater than 2 
%	events
%	\item How feasible is this representation
%	\item Trying to find the sweet spot before the combinatorial explosion
%	\item Bounded superposition approach at string level
%	\item Using a finite state automaton (looking at all strings at once)
%	\item Hybrid approach
%	\item Added constraints before superposition
%\end{itemize}
%%%%%%%%%%%%%%%%%%%%%%%%%%%%%%%%%%%%%%%%%%%%%%%%%%%%%%%%%%%%%%%%%%%%%%%%%%%%%%%%
%                                                                              %
%                            Actual Document Begins                            %
%                                                                              %
%%%%%%%%%%%%%%%%%%%%%%%%%%%%%%%%%%%%%%%%%%%%%%%%%%%%%%%%%%%%%%%%%%%%%%%%%%%%%%%%
\section{Introduction}
%The task of interpreting temporal information in text necessitates a strong 
%representational model.
%
%Text such as 
%
%Extracting 
%
%In modelling this information for computation, there is a usefulness in having 
%a string-based representation, as strings are necessarily finite, while models 
%may be infinite.
%
%We use strings as a basis for representing event data in a computationally 
%tractable way. Strings allow us to bring information together such that it is 
%both human-readable and digitally-operable.
%
%In contrast to the models of predicate logic, which may be infinite, strings 
%are useful both as models and as computational entities.

This paper explores using string-based models to effectively represent event 
data such as might be found in a document annotated with TIMEML. We describe 
how such data may be simply translated to strings, as well as operations on the 
these resultant strings which may be used to infer new information.

We believe strings pose a strong candidate for modelling this information over 
other approaches such as models based in predicate logic, which may inform an 
infinite number of possibilities, whereas strings are computational entities, 
and are often simpler to handle, being able to operate on the level of 
individual strings, which would be impossible using other models.

An \textit{event-string} is a string $s \in \Sigma^+$ for an alphabet 
$\Sigma$, the powerset of some fixed set $A$ of fluents \textit{i.e.} 
predicates which have an associated temporality. If the 
string $s = \alpha_1\cdots\alpha_n$ is taken as a sequence of $n$ moments in 
time, each symbol $\alpha_i \in s$ represents the set of relevant fluents 
holding at the 
moment indexed by $i$. The string is read from left to right chronologically, 
so that any predicates which hold at the moment at index $i$ are understood to 
have held before the moment indexed by $j$ if and only if $i < j$.

The precise duration of each moment is taken as unimportant in the current 
discussion, 
and thus the strings model an inertial world, whereby \textit{change} is the 
only mark of progression from one moment to the next -- as Aristotle said, 
``there is no time 
without change".

Thus, if $\alpha_i = \alpha_{i+1}$ for any 
$1 \leq i < n$, then either symbol may be safely deleted from $s$ without 
affecting the interpretation of the string, as the remaining symbol is simply 
taken as representing a longer moment. This operation of removing 
repetition from 
the event-string is known as \textit{block-compression}. We may also describe 
the inverse of this process, which introduces repeated elements in an 
event-string without changing its interpretation, allowing us greater 
flexibility in our manipulation of the string. These operations are detailed 
in the next section.

\section{Superposition and Block Compression}
With two strings $s$ and $s'$ of the same length $n$
built from an alphabet $\Sigma$, the powerset of some fixed set $A$,
the \textit{superposition} $s ~\&~ s'$ of $s$ and $s'$ is their componentwise 
union:
\begin{align*}
\alpha_1\cdots\alpha_n \ \&\ 
\alpha'_1\cdots\alpha_n' & \ :=\
(\alpha_1\cup\alpha'_1)\cdots(\alpha_n\cup\alpha'_n)
\end{align*}
For convenience of notation, we will use boxes rather than curly braces 
$\{$~$\}$ to represent sets in $\Sigma$, such that each symbol $\alpha$ in a 
string $s$ corresponds to exactly one box. For example, with $a, b, c, d \in A$:
\begin{align*}
\fbox{$a\vph{,'}$}\fbox{$c\vph{,'}$} \ \& \ 
\fbox{$b\vph{,'}$}\fbox{$d\vph{,'}$} \ = \
\fbox{$a, b\vph{,'}$}\fbox{$c, d\vph{,'}$} \ \in\ \Sigma^2
\end{align*}
Extending this function to languages $L$ and $L'$ over the same alphabet is a 
simple matter of 
collecting the superpositions of strings from each language of equal length: 
\begin{align*}
L ~\&~ L' & \ :=\ \bigcup_{n\geq 0}
\{s~\&~s'\ | \ s\in L\cap \Sigma^n\mbox{ and }s'\in L'\cap \Sigma^n\}
\end{align*}
For example, $L ~\&~ \fbox{\vph{'}}^\ast = L$.
If $L$ and $L'$ are regular languages computed by finite automata
with transitions $\to$ and $\to'$, then the superposition $L\& L'$ is
a regular language computed by a finite automaton with transitions
$\Rightarrow$
formed by running $\to$ and $\to'$ in lockstep
according to the rule 
\begin{align*}
\frac{q \sta{\alpha}{\to} r   \hspace{.4in} q' \sta{\alpha'}{\to'} r'}{
	(q,q') \sta{\alpha\cup\alpha'}{\Rightarrow} (r,r')}
\end{align*}
The disadvantage of this process is that it requires the strings to be of equal 
length, which is an idealised situation. In order to work around this, we may 
manipulate the strings in such a way that they conform to our needs. For 
example, we 
can cause a string $s=\alpha_1\cdots\alpha_n$ to \textit{stutter} such that 
$\alpha_i=\alpha_i+1$ for some integer $0 < i < n$. For example, 
\fbox{$a\vph{'}$}\fbox{$a\vph{'}$}\fbox{$a\vph{'}$}\fbox{$c\vph{'}$}\fbox{$c\vph{'}$}
 is a stuttering version of 
\fbox{$a\vph{'}$}\fbox{$c\vph{'}$}. If 
a string does not 
stutter, it is \textit{stutterless}, and we can transform a stuttering string 
to this 
state by using ``block compression":
\begin{align*}
\bc(s)  &\ \ :=\ \
\left\{ \begin{array}{ll}
s & \mbox{ if }~length(s)\leq 1 \\
\bc({\alpha}s')  & \mbox{ if } s={\alpha}{\alpha}s'\\
\alpha ~{\bc}({\alpha'}{s'})  
& \mbox{ if } s=\alpha{\alpha'}{s'} \mbox{ with } \alpha\neq\alpha'
\end{array}
\right. 
\end{align*}
This function can be applied multiple times to a string, but the output will 
not change after the first application: $\bc(\bc(s)) = \bc(s)$. We can also use 
the inverse of this function to generate infinitely many stuttering strings:
\begin{align*}
\bc^{-1}(\fbox{$a\vph{,'}$}\fbox{$c\vph{,'}$}) & = 
\fbox{$a\vph{,'}$}\fbox{$a\vph{,'}$}\fbox{$c\vph{,'}$} \\
& = \fbox{$a\vph{,'}$}\fbox{$c\vph{,'}$}\fbox{$c\vph{,'}$} \\
& = \fbox{$a\vph{,'}$}\fbox{$a\vph{,'}$}\fbox{$c\vph{,'}$}\fbox{$c\vph{,'}$} \\
& \hspace{3em} \vdots
\end{align*}
We can say that any of the strings generated by this inverse block compression 
are \textit{$\bc{}$-equivalent}. Precisely, a string $s'$ is $\bc{}$-equivalent 
to a 
string 
$s$ iff $s' \in \bc^{-1}\bc(s)$.

We can now define the \textit{asynchronous superposition} of strings $s$ and 
$s'$ as the (provably) \textit{finite} set obtained by block compressing the 
\textit{infinite} 
language generated by superposing the strings which are $\bc$-equivalent to $s$ 
and $s'$:
\begin{align*}
s\ \&_{\bc} \ s' & \ := \
\{\bc(s'') \ | \ s''\in \bc^{-1}\bc(s)\ \& \ \bc^{-1}\bc(s')\}
\end{align*}
For example \fbox{$a\vph{,'}$}\fbox{$c\vph{,'}$} $\&_{\bc}$ 
\fbox{$b\vph{,'}$}\fbox{$d\vph{,'}$} will comprise 
three strings:
\begin{align*}
\fbox{$a, b\vph{,'}$}\fbox{$c, d\vph{,'}$}\\
\fbox{$a, b\vph{,'}$}\fbox{$a, d\vph{,'}$}\fbox{$c, d\vph{,'}$}\\
\fbox{$a, b\vph{,'}$}\fbox{$b, c\vph{,'}$}\fbox{$c, d\vph{,'}$}\\
\end{align*}
In order to avoid generating all possible strings when using the inverse block 
compression, we introduce an upper bound to the length of the strings which 
will be superposed. It can be shown that with two strings of length $n$ and 
$n'$, the longest $\bc$-unique string (one which has no shorter 
$\bc$-equivalent strings) produced through asynchronous superposition will be 
of length $n+n'-1$.

\section{Upper Bound on Asynchronous Superposition}
\paragraph{Proposition 1.} {\sl For all  $s, s' \in \Sigma^+$
and all $s''\in s\ \hat{\&}\ s'$,}
\begin{align*}
length(s'')\ \leq\ length(s) + length(s') -1
\end{align*}
\noindent
For all $s,s'\in \Sigma^{\ast}$, we define a finite set
$s~\hat{\&}~s'$ of strings over $\Sigma$ with enough of 
the strings in $\bc^{-1}\bc(s)~\&~\bc^{-1}\bc(s')$
to form $s~\&_{\bc}~s'$.
The definition proceeds by induction on $s$ and $s'$, with
\begin{align*}
\epsilon\ \hat{\&}\ \epsilon \ :=& \ \{\epsilon\}\\
\epsilon\ \hat{\&}\ s \ :=& \ \emptyset\ \ \mbox{ for } s\neq\epsilon\\
s\ \hat{\&}\ \epsilon \ :=& \ \emptyset\ \ \mbox{ for } s\neq\epsilon
\end{align*}
and for all $\alpha,\alpha'\in \Sigma$,
\begin{align*}
\alpha s~\hat{\&}~\alpha's' \ :=& \ 
(\alpha\cup \alpha')(\alpha s\ \hat{\&}\ s'
\ | \ s~\hat{\&}~\alpha's' \ | \  s~\hat{\&}~s')\\
=&~\{(\alpha\cup\alpha')s''\ | \ s''\in (\alpha s~\hat{\&}~s') \cup 
(s~\hat{\&}~\alpha's') \cup (s~\hat{\&}~s')\}
\end{align*}
Note that even if neither $s$ nor $s'$ stutters,
a string in $s\ \hat{\&}\ s'$ may stutter
(e.g., $\fbox{$a,c\vph{,'}$}\fbox{$a,c\vph{,'}$}\in
\fbox{$a\vph{,'}$}\fbox{$c\vph{,'}$}\ \hat{\&}\ 
\fbox{$c\vph{,'}$}\fbox{$a\vph{,'}$}$)
but can be made stutterless through $\bc$.

\paragraph{Proposition 2.} {\sl For all $s,s'\in \Sigma^+$,}
\begin{align*}
s\ \hat{\&}\ s' & \ \subset\ \bc^{-1}\bc(s)\ \&\ \bc^{-1}\bc(s')
\end{align*}
and
\begin{align*}
\{\bc(s'')\ | \ s''\in s\ \hat{\&}\ s'\}
& \ = \   s\ \&_{\bc}\ s'
\end{align*}
Now, for any integer $k > 0$ and string $s = \alpha_1\cdots\alpha_n$ over 
$\Sigma$, we introduce a new 
function 
$pad_k$ which will generate the set of strings with length $k$ which are 
$\bc$-equivalent to $s$:
%For $s=\alpha_1\cdots\alpha_n$,
%let $\rr_k(\alpha_1\cdots\alpha_n)$ consist of strings 
%in $\alpha_1^+\cdots\alpha_n^+$ of length $k$ 
\begin{align*}
pad_k(\alpha_1\cdots\alpha_n)~:=&~~\alpha_1^+\cdots\alpha_n^+\ \cap\ \Sigma^k \\
=&~~\{
\alpha_1^{k_1}\cdots\alpha_n^{k_n}\ | \
k_1,\ldots,k_n\geq 1
\mbox{ and } \sum_{i=1}^n k_i = k \}\\
\subset&~~\bc^{-1}\bc(\alpha_1\cdots\alpha_n)
\end{align*}
For example, $pad_{4}(\fbox{$a\vph{'}$}\fbox{$c\vph{'}$})$ will 
generate \{
\fbox{$a\vph{'}$}\fbox{$a\vph{'}$}\fbox{$a\vph{'}$}\fbox{$c\vph{'}$}, 
\fbox{$a\vph{'}$}\fbox{$a\vph{'}$}\fbox{$c\vph{'}$}\fbox{$c\vph{'}$}, 
\fbox{$a\vph{'}$}\fbox{$c\vph{'}$}\fbox{$c\vph{'}$}\fbox{$c\vph{'}$} \}. We 
can use this new function in our 
calculation of asynchronous superposition, to limit the generation of strings 
from the inverse block compression step. Since we know from Proposition 1 that 
the maximum possible 
length we might need is $n+n'-1$, we can use this value in the $pad$ function 
to just 
generate the strings of that length, giving us a new definition of asynchronous 
superposition:
\paragraph{Corollary 3.} {\sl For any $s,s'\in \Sigma^+$
	with nonzero lengths $n$ and $n'$ respectively,}
\begin{align*}
s \ \&_{\bc} \ s' & \ = \
\{\bc(s'')\ | \ s''\in pad_{n+n'-1}(s)\ \& \ pad_{n+n'-1}(s')\}
\end{align*}
Neither $s\ \hat{\&}\ s'$ nor $pad_{n+n'-1}(s)\ \& \ pad_{n+n'-1}(s')$
need be a subset of the other, even though,
under the assumptions of Corollary 3, 
both sets block compress to $s \ \&_{\bc} \ s'$.

\section{Event Representation}
Now we may use asynchronous superposition to generate the 13 strings in 
\fbox{\vph{$e'$}}\fbox{$e\vph{e'}$}\fbox{\vph{$e'$}} 
$\&_{\bc}$ 
\fbox{\vph{$e'$}}\fbox{$e'$}\fbox{\vph{$e'$}}, each of which corresponds to one 
of the unique interval relations in \cite{allen1983maintaining}.
We use the empty box \fbox{$\vph{'}$} as a string of length 1 (not to confused 
with the empty string $\epsilon$, which is length 0) to bound events, allowing 
us to represent the fact that they are finite -- they have a beginning and 
ending point. It is prudent to assume that we will deal only with finite event 
data, such that there are no fluents which do not have both an associated 
start-point and end-point. If such a fluent were to occur, it would be trivial 
to add it to every position in the string. 

The bounding boxes represent the time 
before and after the event occurs, during which no other fluents belonging to 
the alphabet $\Sigma$ 
hold. The 
Allen Relation Strings are laid out 
below:
\begin{center}
	\begin{tabular}{ l@{\hskip 1in}c@{\hskip 1in}r }
		$e~\mathbf{=}~e'$ & \fbox{\vph{$e'\vph{,e'}$}}\fbox{$e, 	
		e'\vph{e'}$}\fbox{\vph{$,e'$}} & equal\\[0.6em]		
		$e~\mathbf{s}~e'$ & \fbox{\vph{$e'\vph{,e'}$}}\fbox{$e, 
		e'\vph{e'}$}\fbox{$e'\vph{,e'}$}\fbox{\vph{$,e'$}} & starts\\[0.6em]
		$e~\mathbf{si}~e'$ & \fbox{\vph{$e'\vph{,e'}$}}\fbox{$e, 		
		e'\vph{e'}$}\fbox{$e\vph{,e'}$}\fbox{\vph{$,e'$}} & 		
		starts~(inverse)\\[0.6em]
		$e~\mathbf{f}~e'$ & \fbox{\vph{$e'\vph{,e'}$}}\fbox{$ 		
		e'\vph{,e'}$}\fbox{$e, e'\vph{,e'}$}\fbox{\vph{$,e'$}} & 		
		finishes\\[0.6em]
		$e~\mathbf{fi}~e'$ & \fbox{\vph{$e'\vph{,e'}$}}\fbox{$ 		
		e\vph{,e'}$}\fbox{$e, e'\vph{,e'}$}\fbox{\vph{$,e'$}} &		
		finishes~(inverse)\\[0.6em]
		$e~\mathbf{d}~e'$ & \fbox{\vph{$,e'$}}\fbox{$ e'\vph{,e'}$}\fbox{$e, 
		e'\vph{,e'}$}\fbox{$ e'\vph{,e'}$}\fbox{\vph{$,e'$}} & during\\[0.6em]
		$e~\mathbf{di}~e'$ & \fbox{\vph{$e'\vph{,e'}$}}\fbox{$ 
		e\vph{,e'}$}\fbox{$e, e'\vph{,e'}$}\fbox{$ 
		e\vph{,e'}$}\fbox{\vph{$,e'$}} & during~(inverse)\\[0.6em]
		$e~\mathbf{o}~e'$ & \fbox{\vph{$,e'$}}\fbox{$ e\vph{,e'}$}\fbox{$e, 
		e'\vph{,e'}$}\fbox{$ e'\vph{,e'}$}\fbox{\vph{$,e'$}} & 
		overlaps\\[0.6em]
		$e~\mathbf{oi}~e'$ & \fbox{\vph{$e'\vph{,e'}$}}\fbox{$ 
		e'\vph{,e'}$}\fbox{$e, e'\vph{,e'}$}\fbox{$ 
		e\vph{,e'}$}\fbox{\vph{$,e'$}} & overlaps~(inverse)\\[0.6em]
		$e~\mathbf{m}~e'$ & \fbox{\vph{$,e'$}}\fbox{$ e\vph{,e'}$}\fbox{$ 
		e'\vph{,e'}$}\fbox{\vph{$,e'$}} & meets\\[0.6em]
		$e~\mathbf{mi}~e'$ & \fbox{\vph{$e'\vph{,e'}$}}\fbox{$ 
		e'\vph{,e'}$}\fbox{$ e\vph{,e'}$}\fbox{\vph{$,e'$}} & 
		meets~(inverse)\\[0.6em]
		$e~\mathbf{<}~e'$ & \fbox{\vph{$,e'$}}\fbox{$ 
		e\vph{,e'}$}\fbox{$\vph{,e'}$}\fbox{$ e'\vph{,e'}$}\fbox{\vph{$,e'$}} & 
		before\\[0.6em]
		$e~\mathbf{>}~e'$ & \fbox{\vph{$e'\vph{,e'}$}}\fbox{$ 
		e'\vph{,e'}$}\fbox{$\vph{,e'}$}\fbox{$ e\vph{,e'}$}\fbox{\vph{$,e'$}} & 
		after
	\end{tabular}
\end{center}
These Allen Relations are included in the attributes of TIMEML, the standard 
markup language used for event annotation in texts, as TLINKs. By extracting 
the TLINKs from an annotated document, and translating them to our event-string 
representation, we may begin to reason about the relationships between 
annotated events which do not have an associated TLINK in the markup. For 
example, the document may give us a relation between events $e$ and $e'$, and 
another relation between $e'$ and $e''$, and from this we may infer the 
possible 
relations between $e$ and $e''$.

% Do I need to prove this?
% Define "well-formed event-string"
As asynchronous superposition is a commutative, associative, homomorphic binary 
relation over \iffalse{}well-formed\fi{}event-strings, we may superpose 
arbitrary numbers of event-strings. Note, however, that superposing even a 
small number of unconstrained, bounded events leads to a massive blowup in the 
number of possible outcomes. Currently, attempting to compute the result of 
more than five is beyond the capabilities of available hardware:
\begin{align*}
2 ~\mbox{bounded events} &\to 13 ~\mbox{outcomes}\\
3 ~\mbox{bounded events} &\to 409 ~\mbox{outcomes}\\
4 ~\mbox{bounded events} &\to 23917 ~\mbox{outcomes}\\
5 ~\mbox{bounded events} &\to 2244361 ~\mbox{outcomes}
\end{align*}
Clearly, simply superposing bounded events in this manner is infeasible, as it 
is unreasonable to expect that any given document would contain five or less 
events. In order to avoid generating such a large number of computed 
event-strings, it is necessary to add constraints where appropriate as to what 
strings may be considered allowable for a particular context.

\section{Constraints on Event-Strings}
Two approaches to constraints may be implemented, which are not mutually 
exclusive. The first is to prevent unwanted strings from being generated, based 
on the nature of the operand strings, and the second is to remove disallowed 
strings from the set of outputs. The former approach is preferred from a 
computational standpoint, as there is less data to store and process.

We assume that every fluent we encounter has exactly one beginning and one 
ending -- that is, that events do not \textit{resume} once they have ended. 
Events of the same type may stop and start frequently, but by assuming that 
every instance of an event will have a uniquely identifying fluent, we can 
discard any strings which feature such a resumption. We also have the 
constraint that fluents may be referred to multiple times

% Isolated fluents
% Reconciled fluents
% Disallowed sequences

%%%%%%%%%%%%%%%%%%%%%%%%%%%%%%%%%%%%%%%%%%%%%%%%%%%%%%%%%%%%%%%%%%%%%%%%%%%%%%%%
%                                                                              %
%                             Actual Document Ends                             %
%                                                                              %
%%%%%%%%%%%%%%%%%%%%%%%%%%%%%%%%%%%%%%%%%%%%%%%%%%%%%%%%%%%%%%%%%%%%%%%%%%%%%%%%
\bibliographystyle{chicago}{}
\bibliography{mybib}
\end{document}