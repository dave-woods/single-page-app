\documentclass[a4paper,11pt,leqno]{article}

\usepackage{graphicx}  %%% for including graphics
\usepackage{url}       %%% for including URLs
\usepackage{times}
\usepackage{amssymb,amsmath,amscd}
\usepackage{natbib}
\usepackage[margin=25mm]{geometry}
\usepackage{fancyvrb}
\usepackage[hang,flushmargin]{footmisc}
\usepackage{makecell}

% Tim's custom commands
\newcommand{\bc}{{\rm b\!c}}
\newcommand{\unpad}{\mbox{{\rm unpad}}}
\newcommand{\vph}[1]{\vphantom{#1}}
\newcommand{\sta}[2]{\stackrel{#1}{#2}}

% David's custom commands
\newcommand{\ebox}[1]{\fbox{$\vph{',}#1$}}
\newcommand{\eboxl}[1]{\fbox{$\vph{'}#1$}}
\newcommand{\eboxh}[1]{\fbox{$\vph{,}#1$}}
\newcommand{\eboxb}[1]{\fbox{$\vph{@}#1$}}

\title{Towards Efficient String Processing of Annotated Events}
\date{}

\author{David Woods\\
	ADAPT Centre\\
	Trinity College Dublin, Ireland\\
	\texttt{dwoods@tcd.ie}
	\and Tim Fernando, Carl Vogel\\
	Computational Linguistics Group \\
	Trinity Centre for Computing and Language Studies\\
	School of Computer Science and Statistics\\
	Trinity College Dublin, Ireland\\
	\texttt{tim.fernando@tcd.ie}, \texttt{vogel@tcd.ie}
}

\begin{document}
\maketitle
\thispagestyle{empty}
\pagestyle{empty}
\begin{abstract}
\noindent
This paper explores the use of strings as models to effectively represent event 
data such as might be found in a document annotated with ISO-TimeML. We 
describe the translation of such data to strings, as well as a number of 
operations, such as superposition, which may be used to manipulate these 
strings in order to infer new information. Some advantages and limitations of 
the operations are discussed, including issues of over-generation, which can be 
mitigated though the use of suitable constraints. In particular, we look at how 
Allen Relations, which might be extracted from a document annotated with 
ISO-TimeML, can be understood as useful constraints, and translated to strings.
\end{abstract}
%%%%%%%%%%%%%%%%%%%%%%%%%%%%%%%%%%%%%%%%%%%%%%%%%%%%%%%%%%%%%%%%%%%%%%%%%%%%%%%%
%                                                                              %
%                            Actual Document Begins                            %
%                                                                              %
%%%%%%%%%%%%%%%%%%%%%%%%%%%%%%%%%%%%%%%%%%%%%%%%%%%%%%%%%%%%%%%%%%%%%%%%%%%%%%%%
\section{Introduction}\label{intro}
This paper explores the use of strings as models to effectively represent event 
data such as might be found in a document annotated with ISO-TimeML. It is 
described how such data may be simply translated to strings, and how to infer 
information through operations on these strings. Strings are basic 
computational entities that can be more readily manipulated by machines than 
the infinite models of predicate logic. Finite sets of strings serve as finite 
models.

We fix a finite set $A$ of fluents (temporal propositions), 
and encode sets of these fluents as symbols to allow any number of them to hold 
at a time (as in \citealp{fernando2016regular}). A string $s=\alpha_1 \cdots 
\alpha_n$ of subsets $\alpha_i$ of $A$ can be construed as a finite model 
consisting of $n$ moments of time $i \in \{1, \ldots, n\}$ with $\alpha_i$ 
specifying all fluents (in $A$) that (as unary predicates) hold simultaneously 
at $i$.

Throughout this paper, a fluent $a \in A$ will be understood as naming 
an event, and the powerset $2^{A}$ of $A$ will serve as an alphabet $\Sigma = 
2^{A}$ of an \textit{event-string} $s \in \Sigma^+$. Such strings are finite 
models of Monadic Second Order logic, and are amenable to finite state methods. 
We will further restrict them in Section \ref{constraints}, with a focus on 
using Allen Relations adopted in ISO-TimeML, in order to analyse inference over 
a finite search space.

An event-string $\alpha_1 \cdots \alpha_n$ is read from left to right 
chronologically, 
so that any predicates which hold at the moment at index $i$ are understood to 
have held before another moment indexed by $j$ if and only if $i < j$. The 
precise duration of each moment is taken as unimportant in the current 
discussion, 
and thus the strings model an inertial world, whereby \textit{change} is the 
only mark of progression from one moment to the next -- ``But neither does time 
exist without change" (Aristotle, \textit{Physics IV}). Thus, if $\alpha_i = 
\alpha_{i+1}$ for any 
$1 \leq i < n$, then either $\alpha_i$ or $\alpha_{i+1}$ may be safely deleted 
from $s$ without 
affecting the interpretation of the string, as the remaining symbol is simply 
taken as representing a longer moment. This operation of removing 
repetition from 
the event-string is known as \textit{block compression} 
\citep{fernando2016regular}. The inverse of this process introduces repeated 
elements in an event-string for greater flexibility in manipulating strings. 
These operations are detailed in Section \ref{sp and bc}.

We see that strings may provide useful finite models for event data, once 
sufficiently constrained. This is in order to avoid a large combinatorial 
blow-up when reconciling information from different strings.

In the following section, we see an example from ISO-TimeML which we might 
convert to an event-string, and why it would be useful to do so. Sections 
\ref{sp and bc} and \ref{upper} provide the formal string operations used to 
manipulate and combine information from event-strings. How these operations are 
useful in regards to Allen Relations is described in Section \ref{event rep}, 
and Section \ref{constraints} explores constraints to prevent malformed 
event-strings. Finally, Section \ref{application} shows how the framework can 
be 
applied to data from an ISO-TimeML document.

\section{Motivation in ISO-TimeML}\label{motivation}
ISO-TimeML \citep{pustejovsky2010iso} is a standard markup language used for 
the annotation of events (and their interrelations) in texts. Of particular 
interest to us here are the TLINK elements, which indicate the relations 
between pairs of fluents found in the document. Though not every fluent will 
necessarily be linked with another in this manner, a majority will be. The 
TIMEBANK Corpus \citep{pustejovsky2003timebank} provides a large number of 
documents annotated using the 1.1 TimeML standard (a predecessor to 
ISO-TimeML), which we may extract TLINKs from.

According to the markup specification \citep{timeml2005timeml}, a TLINK is 
required to have the following attributes: either a \texttt{timeID} or 
\texttt{eventInstanceID} attribute, referring to some fluent in the text, as 
well as either a \texttt{relatedToTime} or \texttt{relatedToEventInstance} 
attribute, which will refer to another fluent, and also a \texttt{relType} 
attribute, declaring the relation between the two fluents. Other attributes are 
optional and not relevant to the current discussion.

In order to give a more concrete understanding, let us take a small fragment 
from an ISO-TimeML document, which will give us three TLINK nodes:
\begin{subequations}
\begin{align}
\label{example:tlink1}\mbox{\texttt{<TLINK lid="l2" relType="IS\_INCLUDED" 
		eventInstanceId="ei1"}}\\ 
\mbox{\texttt{relatedToTime="t1" origin="USER"/>}} \notag\\
\label{example:tlink2}\mbox{\texttt{<TLINK lid="l9" relType="IS\_INCLUDED" 
		timeID="t1"}}\\ 
\mbox{\texttt{relatedToEventInstance="ei9" origin="USER"/>}} \notag\\
\label{example:tlink3}\mbox{\texttt{<TLINK lid="l10" relType="BEFORE" 
		eventInstanceID="ei9"}}\\
\mbox{\texttt{relatedToEventInstance="ei10" origin="USER"/>}} \notag
\end{align}
\end{subequations}
The value of \texttt{relType} will correspond to exactly one of the relations 
described in \cite{allen1983maintaining}, though it should be noted that some 
of these Allen Relations can correspond to multiple \texttt{relType}s (for 
example, \texttt{IDENTITY} and \texttt{SIMULTANEOUS} are both covered by the 
Allen Relation \textit{equal}). The other attributes (\texttt{lid}, 
\texttt{origin}) in each TLINK may be ignored for now.

We can represent the information in 
(\ref{example:tlink1})--(\ref{example:tlink3}) in predicate logic as 
follows:
\begin{subequations}
	\begin{align}
	&\mbox{\textit{Includes}}(t1,ei1)\\
	&\mbox{\textit{Includes}}(ei9,t1)\\
	&\mbox{\textit{Before}}(ei9,ei10)
	\end{align}
\end{subequations}
where \textit{Includes} (the inverse of \texttt{IS\_INCLUDED}) and 
\textit{Before} are binary 
relations, corresponding to the Allen Relations \textit{during (inverse)} and 
\textit{before}, respectively (see Section \ref{event rep}).

An issue with this representation is that the full picture of the  
chronological sequence of events is not intuitively obvious from the three 
predicates. It is possible to create a set of inference rules (see 
\citealp{setzer2005role}) which allow for drawing new conclusions from the 
given information -- for example, that the predicate \textit{Before}$(t1, 
ei10)$ also 
holds. However, the amount of information that can be obtained from any single 
binary relation is relatively small, and reasoning about more than two fluents 
at once can require the conjunction of multiple relations.
 
Using an event-string, we may include all of the above information in a single, 
readable string, which allows us to reason about the relations between any 
number of fluents. Additionally, we provide facilities to infer new relations 
from this event-string (see Section \ref{constraints}).

\section{Superposition and Block Compression}\label{sp and bc}
In order to usefully collect information from multiple strings into a single 
string, we define here the operation of \textit{superposition}.
With two strings $s$ and $s'$ of the same length $n$
built from an alphabet $\Sigma$, the powerset of some fixed set 
$A$,
the superposition $s ~\&~ s'$ of $s$ and $s'$ is their componentwise 
union:
\begin{align}
\alpha_1\cdots\alpha_n \ \&\ 
\alpha'_1\cdots\alpha_n' & \ :=\
(\alpha_1\cup\alpha'_1)\cdots(\alpha_n\cup\alpha'_n)
\end{align}
For convenience of notation, we will use boxes rather than curly braces 
$\{$~$\}$ to represent sets in $\Sigma$, such that each symbol $\alpha$ in a 
string $s$ corresponds to exactly one box. For example, with $a, b, c, d \in 
A$:
\begin{align}
\ebox{a}\ebox{c} \ \& \ \ebox{b}\ebox{d} \ = \
\ebox{a, b}\ebox{c, d} \ \in\ \Sigma^2
\end{align}
Extending superposition to languages $L$ and $L'$ over the same alphabet is a 
simple matter of 
collecting the superpositions of strings of equal length from each language: 
\begin{align}
L ~\&~ L' & \ :=\ \bigcup_{n\geq 0}
\{s~\&~s'\ | \ s\in L\cap \Sigma^n\mbox{ and }s'\in L'\cap \Sigma^n\}
\end{align}%$^{^{\mbox{$\ast$}}} = s$
For example, $L ~\&~ \eboxb{}^{\mbox{$\ast$}} = L$.
If $L$ and $L'$ are regular languages computed by finite automata
with transitions $\to$ and $\to'$, then the superposition $L~\&~ L'$ is
a regular language computed by a finite automaton with transitions
$\Rightarrow$
formed by running $\to$ and $\to'$ in lockstep
according to the rule 
\begin{align}
\frac{q \sta{\alpha}{\to} r   \hspace{.4in} q' \sta{\alpha'}{\to'} r'}{
	(q,q') \sta{\alpha\cup\alpha'}{\Rightarrow} (r,r')}
\end{align}
A disadvantage of this operation is that it requires the string operands to be 
of equal length, which is an overly specific case. In order to generalise this 
procedure to strings of arbitrary lengths, we may 
manipulate the strings to move away from the synchrony of the lockstep 
procedure. One such manipulation is that 
we can cause a string $s=\alpha_1\cdots\alpha_n$ to \textit{stutter} such that 
$\alpha_i=\alpha_{i+1}$ for some integer $0 < i < n$. For example, 
\eboxl{a}\eboxl{a}\eboxl{a}\eboxl{c}\eboxl{c}
 is a stuttering version of 
\eboxl{a}\eboxl{c}. If 
a string does not 
stutter, it is \textit{stutterless}, and we can transform a stuttering string 
to this 
state by using ``block compression":
\begin{align}
\bc(s)  &\ \ :=\ \
\left\{ \begin{array}{ll}
s & \mbox{ if \textit{length}}(s)\leq 1 \\
\bc({\alpha}s')  & \mbox{ if } s={\alpha}{\alpha}s'\\
\alpha ~{\bc}({\alpha'}{s'})  
& \mbox{ if } s=\alpha{\alpha'}{s'} \mbox{ with } \alpha\neq\alpha'
\end{array}
\right. 
\end{align}
This function can be applied multiple times to a string, but the output will 
not change after the first application: $\bc(\bc(s)) = \bc(s)$. We can also use 
the inverse of this function to generate infinitely many stuttering strings:
\begin{align}
\bc^{-1}(\ebox{a}\ebox{c}) = 
\{ \ebox{a}\ebox{c},
\ebox{a}\ebox{a}\ebox{c},
\ebox{a}\ebox{c}\ebox{c}, 
\ebox{a}\ebox{a}\ebox{c}\ebox{c}, 
\ldots \}
\end{align}
We can say that any of the strings generated by this inverse block compression 
are \textit{$\bc{}$-equivalent}. Precisely, a string $s'$ is $\bc{}$-equivalent 
to a 
string 
$s$ iff $s' \in \bc^{-1}\bc(s)$.

We can now define the \textit{asynchronous superposition} $s~\&_*~s'$ of 
strings $s$ and 
$s'$ as the (provably) \textit{finite} set obtained by block compressing the 
\textit{infinite} 
language generated by superposing the strings which are $\bc$-equivalent to $s$ 
and $s'$:
\begin{align}
s\ \&_* \ s' & \ := \
\{\bc(s'') \ | \ s''\in \bc^{-1}\bc(s)\ \& \ \bc^{-1}\bc(s')\}
\end{align}
For example, \ebox{a}\ebox{c} $\&_*$ 
\ebox{b}\ebox{d} will comprise 
three strings:
\begin{align}
\{ \ebox{a, b}\ebox{c, d},
\ebox{a, b}\ebox{a, d}\ebox{c, d},
\ebox{a, b}\ebox{b, c}\ebox{c, d} \}
\end{align}
In order to avoid generating all possible strings when using the inverse block 
compression, we introduce an upper bound to the length of the strings which 
will be superposed. It can be shown that with two strings of length $n$ and 
$n'$, the longest $\bc$-unique string (one which has no shorter 
$\bc$-equivalent strings) produced through asynchronous superposition will be 
of length $n+n'-1$.

\section{Upper Bound on Asynchronous Superposition}\label{upper}
The aforementioned bound on the length of superposed strings can be established 
as follows.
For all $s,s'\in \Sigma^{\ast}$, we define a finite set
$s~\hat{\&}~s'$ of strings over $\Sigma$ with enough of 
the strings in $\bc^{-1}\bc(s)~\&~\bc^{-1}\bc(s')$
to form $s~\&_*~s'$.
The definition proceeds by induction on $s$ and $s'$, with
\begin{subequations}
\begin{align}
\epsilon\ \hat{\&}\ \epsilon \ :=& \ \{\epsilon\}\\
\epsilon\ \hat{\&}\ s \ :=& \ \emptyset\ \ \mbox{ for } s\neq\epsilon\\
s\ \hat{\&}\ \epsilon \ :=& \ \emptyset\ \ \mbox{ for } s\neq\epsilon
\end{align}
\end{subequations}
and for all $\alpha,\alpha'\in \Sigma$,
\begin{align}
\alpha s~\hat{\&}~\alpha's' \ :=&~\{(\alpha\cup\alpha')s''\ | \ s''\in (\alpha 
s~\hat{\&}~s') \cup 
(s~\hat{\&}~\alpha's') \cup (s~\hat{\&}~s')\}
\end{align}
Note that a string in $s\ \hat{\&}\ s'$ might stutter, even if neither of the 
operands $s$ or $s'$ do
(\textit{e.g.} $\ebox{a,c}\ebox{a,c}\in
\ebox{a}\ebox{c}\ \hat{\&}\ 
\ebox{c}\ebox{a}$). However, it can be made stutterless through block 
compression.
\paragraph{Proposition 1.} {\sl For all  $s, s' \in \Sigma^+$
	and all $s''\in s\ \hat{\&}\ s'$,}
\begin{align}
\mbox{\textit{length}}(s'')\ \leq\ \mbox{\textit{length}}(s) + 
\mbox{\textit{length}}(s') -1
\end{align}
\paragraph{Proposition 2.} {\sl For all $s,s'\in \Sigma^+$,}
\begin{align}
s\ \hat{\&}\ s' & \ \subset\ \bc^{-1}\bc(s)\ \&\ \bc^{-1}\bc(s')
\end{align}
and
\begin{align}
\{\bc(s'')\ | \ s''\in s\ \hat{\&}\ s'\}
& \ = \   s\ \&_*\ s'
\end{align}
Now, for any integer $k > 0$ and string $s = \alpha_1\cdots\alpha_n$ over 
$\Sigma$, we introduce a new 
function 
$pad_k$ which will generate the set of strings with length $k$ which are 
$\bc$-equivalent to $s$:
\begin{subequations}
\begin{align}
pad_k(\alpha_1\cdots\alpha_n)~:=&~~\alpha_1^+\cdots\alpha_n^+\ \cap\ \Sigma^k \\
=&~~\{
\alpha_1^{k_1}\cdots\alpha_n^{k_n}\ | \
k_1,\ldots,k_n\geq 1
\mbox{ and } \sum_{i=1}^n k_i = k \}\\
\subset&~~\bc^{-1}\bc(\alpha_1\cdots\alpha_n)
\end{align}
\end{subequations}
For example, $pad_{4}(\eboxl{a}\eboxl{c})$ will 
generate \{\eboxl{a}\eboxl{a}\eboxl{a}\eboxl{c}, 
\eboxl{a}\eboxl{a}\eboxl{c}\eboxl{c}, 
\eboxl{a}\eboxl{c}\eboxl{c}\eboxl{c}\}. We 
can use this new function in our 
calculation of asynchronous superposition, to limit the generation of strings 
from the inverse block compression step. Since we know from Proposition 1 that 
the maximum possible 
length we might need is $n+n'-1$, we can use this value in the $pad$ function 
to just 
generate the strings of that length, giving us a new definition of asynchronous 
superposition:
\paragraph{Corollary 3.} {\sl For any $s,s'\in \Sigma^+$
	with nonzero lengths $n$ and $n'$ respectively,}
\begin{align}
s \ \&_* \ s' & \ = \
\{\bc(s'')\ | \ s''\in pad_{n+n'-1}(s)\ \& \ pad_{n+n'-1}(s')\}
\end{align}
Neither $s\ \hat{\&}\ s'$ nor $pad_{n+n'-1}(s)\ \& \ pad_{n+n'-1}(s')$
need be a subset of the other, even though,
under the assumptions of Corollary 3, 
both sets block compress to $s \ \&_* \ s'$.

\section{Event Representation}\label{event rep}
Now we may use asynchronous superposition to generate the 13 strings in 
\eboxl{}\eboxl{e}\eboxl{} 
$\&_*$ 
\eboxl{}\eboxl{e'}\eboxl{}, each of which corresponds to one 
of the unique interval relations in \cite{allen1983maintaining}. No more than 
one of these relations may hold between any two fluents, and thus each of the 
13 generated event-strings exists in a distinct possible ``world".
We use the empty box \eboxl{} as a string of length 1 (not to be confused 
with the empty string $\epsilon$, which is length 0) to bound events, allowing 
us to represent the fact that they are finite -- they have a beginning and 
ending point. It is prudent to assume that we will deal only with finite event 
data, such that there are no fluents which do not have both an associated 
start-point and end-point. If such a non-finite fluent without a begining and 
ending were to occur, it could trivially appear in every position in the 
event-string. 

The bounding boxes represent the time 
before and after the event occurs, during which no other fluents $a \in A$ 
are mentioned. The 
event-strings associated with the Allen Relations are laid out 
below:

\begin{center}
	\begin{tabular}{ l@{\hskip 1in}c@{\hskip 1in}r }
		$e~\mathbf{=}~e'$ & \ebox{}\ebox{e, e'}\ebox{} & equal\\[0.4em]		
		$e~\mathbf{s}~e'$ & \ebox{}\ebox{e, e'}\ebox{e'}\ebox{} & 
		starts\\[0.4em]
		$e~\mathbf{si}~e'$ & \ebox{}\ebox{e, e'}\ebox{e}\ebox{} & 		
		starts~(inverse)\\[0.4em]
		$e~\mathbf{f}~e'$ & \ebox{}\ebox{e'}\ebox{e, e'}\ebox{} & 		
		finishes\\[0.4em]
		$e~\mathbf{fi}~e'$ & \ebox{}\ebox{e}\ebox{e, e'}\ebox{} &		
		finishes~(inverse)\\[0.4em]
		$e~\mathbf{d}~e'$ & \ebox{}\ebox{ e'}\ebox{e, e'}\ebox{ e'}\ebox{} & 
		during\\[0.4em]
		$e~\mathbf{di}~e'$ & \ebox{}\ebox{ e}\ebox{e, e'}\ebox{e}\ebox{} & 
		during~(inverse)\\[0.4em]
		$e~\mathbf{o}~e'$ & \ebox{}\ebox{ e}\ebox{e, e'}\ebox{ e'}\ebox{} & 
		overlaps\\[0.4em]
		$e~\mathbf{oi}~e'$ & \ebox{}\ebox{e'}\ebox{e, e'}\ebox{e}\ebox{} & 
		overlaps~(inverse)\\[0.4em]
		$e~\mathbf{m}~e'$ & \ebox{}\ebox{ e}\ebox{e'}\ebox{} & meets\\[0.4em]
		$e~\mathbf{mi}~e'$ & \ebox{}\ebox{e'}\ebox{ e}\ebox{} & 
		meets~(inverse)\\[0.4em]
		$e~\mathbf{<}~e'$ & \ebox{}\ebox{e}\ebox{}\ebox{ e'}\ebox{} & 
		before\\[0.4em]
		$e~\mathbf{>}~e'$ & \ebox{}\ebox{e'}\ebox{}\ebox{ e}\ebox{} & 
		after
	\end{tabular}
\end{center}
These Allen Relations are included in the attributes of ISO-TimeML, as types of 
relation annotated by TLINKs (though some relations are named slightly 
differently). By extracting the TLINKs from an annotated document, and 
translating them to our event-string representation (see Section 
\ref{application}), we may begin to reason about the relationships between 
annotated events which do not have an associated TLINK in the markup. For 
example, the document may give us a relation between events $e$ and $e'$, and 
another relation between $e'$ and $e''$, and from this we may infer the 
possible relations between $e$ and $e''$.

As asynchronous superposition is commutative and associative, we may superpose 
arbitrary numbers of event-strings: $s_1 ~\&_*~ \cdots ~\&_*~ s_n$. We can show 
that superposing $n$ unconstrained bounded event-strings will generate strings 
of maximum length $2n + 1$.\footnote{
\begin{math}
\mbox{The proof is by induction:}\\
\mbox{Let each string to be superposed } s_i \in \{ s_1, \ldots, s_n 
\}~\mbox{be}~
\ebox{}\ebox{e_i}\ebox{}\mbox{, with each }e_i \in A.\\
\mbox{For } n = 2\mbox{: }s_1~\&_*~s_2.\\
\mbox{From Proposition 1, the maximum length 
of the result is } 3 + 3 - 1 = 5 = 2(2) + 1.\\
\mbox{We assume true for }n = p\mbox{, thus the maximum length of }s_1 ~\&_*~ 
\cdots ~\&_*~ s_p\mbox{ is }2(p) + 1.\\
\mbox{Next, we prove for }n = p + 1\mbox{: } s_1 ~\&_*~ \cdots ~\&_*~ s_{p+1} = 
s_1 ~\&_*~ \cdots ~\&_*~ s_p~\&_*~ s_{p+1} = s_{1 \ldots p}~\&_*~s_{p+1}.\\
\mbox{From Proposition 1, the maximum length of the result is }(2(p) + 1) + 3 
- 1 = 2(p + 1) + 1.\\
\mbox{Thus true for }p+1\mbox{, and by induction, true for any }n \geq 2.
\end{math}}
Note, however, that superposing even a relatively small number of 
unconstrained bounded events leads to a massive combinatorial blow-up in the 
number of outcomes, or possible worlds, as each event-string generated from one 
superposition (\textit{e.g.} $s_1 ~\&_*~ s_2$) will in turn be superposed with 
each generated from another (\textit{e.g.} $s_3 ~\&_*~ s_4$). Additionally, 
with each event, the maximum possible length of the strings grows, 
meaning a larger set of strings will be generated at the $pad$ stage. Table 
\ref{table:combinatorics} shows the maximum lengths of the stutterless strings 
generated from $n$ unconstrained bounded events, up to $n = 5$. For $n \geq 2$, 
one can show that $2^n < b(n) < (2n^2-n)^n$, where $b(n)$ represents the number 
of distinct outcomes generated by asynchronously superposing $n$ unconstrained 
bounded 
events.\footnote{A single bounded event may be represented in a (possibly 
stuttering) string of length $2n+1$ in $\sum_{i=1}^{2n-1}i = 2n^2-n$ ways. 
Together, $n$ of these leads to $(2n^2-n)^n$.} Table \ref{table:combinatorics} 
also sets out these numbers.

\noindent
\begin{table}[h!]
	\centering
	\begin{tabular}{| c | c || c | c | c |}
		\hline
		\thead{\textbf{Number of events:}\\$n$} & 
		\thead{\textbf{Maximum length:}\\$2n + 1$} &
		\thead{\textbf{Lower bound:}\\$2^n$} &
		\thead{\textbf{Number $b(n)$ of outcomes from}\\$s_1 ~\&_*~ \cdots 
		~\&_*~ 
		s_n$} &
		\thead{\textbf{Upper bound:}\\$(2n^2 - 1)^n$}
		\\
		\hline
		\thead{$2$} & 
		\thead{$5$} &
		\thead{$4$} &
		\thead{$13$} &
		\thead{$36$}
		\\
		\hline
		\thead{$3$} & 
		\thead{$7$} &
		\thead{$8$} & 
		\thead{$409$} &
		\thead{$3,375$}
		\\
		\hline
		\thead{$4$} & 
		\thead{$9$} &
		\thead{$16$} &
		\thead{$23,917$} &
		\thead{$61,456$}
		\\
		\hline
		\thead{$5$} & 
		\thead{$11$} &
		\thead{$32$} &
		\thead{$2,244,361$} &
		\thead{$184,528,125$}
		\\
		\hline
	\end{tabular}
	\caption{Outcomes of asynchronous superposition of $n$ bounded events}
	\label{table:combinatorics}
\end{table}
Clearly, simply superposing bounded events in this manner is not feasible, 
given the huge growth in the number of possible outcomes, and 
it is unreasonable to expect that any given document should contain five or 
fewer events. In order to avoid generating such a large number of computed 
event-strings, it is necessary to add constraints to limit the
strings that may be considered allowable for a particular context.

Interestingly, because each unconstrained bounded event-string 
\eboxb{}\eboxb{e}\eboxb{} contains exactly one fluent, we may determine the 
maximum possible length of a string generated by superposition, $2n + 1$, from 
the size of the set $A$ of fluents, where $n = |A|$. By keeping track of $|A|$, 
we ensure that the length of the string will always be finite, opening up the 
possibility of using methods from constraint satisfaction, exploiting the 
finite search space.

\section{Constraints on Event-Strings}\label{constraints}
Two approaches to constraints may be implemented, which are not mutually 
exclusive. The first is to prevent unwanted strings from being generated, based 
on the nature of the operand strings, and the second is to remove disallowed 
strings from the set of outputs. The former approach is preferred from a 
computational standpoint, as there is less data to store and process. For 
either, we define some properties of what we may consider to be a 
\textit{well-formed event-string}.

We assume that every fluent we encounter has exactly one beginning and one 
ending -- that is, that events do not \textit{resume} once they have ended. 
Events of the same type may stop and start frequently, but by assuming that 
every instance of an event will have a uniquely identifying fluent, we can 
discard any strings which feature such a resumption.\footnote{We adopt 
simplifying assumptions made in Allen Relations, though it should be noted that 
the distinction between event instances and event types (see 
\citealp{fernando2015semantics}) is not imposed by the 
event-string framework itself, allowing for discontinuous events (such as 
\textit{judder}) in future work.} In this way, fluents are 
\textit{interval-like}. We define the function 
$\rho_{X}$ on strings of sets to component-wise intersect with $X$ for 
any $X 
\subseteq A$ \citep{fernando2016regular}:
\begin{align}
\rho_{X}(\alpha_1 \cdots \alpha_n) := (\alpha_1 \cap X) \cdots 
(\alpha_n 
\cap 
X)
\end{align}
Applying block compression to an event-string which has been reduced with 
$\rho_{\{a\}}$ should produce a single string: \eboxb{}\eboxb{a}\eboxb{}. For 
example, with $a, b \in A$:
\begin{align}
\bc{}(\rho_{\{a\}}(\ebox{}\ebox{a}\ebox{a, b}\ebox{b}\ebox{})) = 
\ebox{}\ebox{a}\ebox{}
\end{align}
Additionally, fluents may be referred to multiple times by different TLINKs in 
an annotated document, and we assume that they will be \textit{consistent} 
within the context of that document \textit{i.e.} if a relation holds between 
$e$ and $e'$, and a relation holds between $e'$ and $e''$, then both instances 
of $e'$ refer to the same fluent. In this case, if a relation also holds 
between $e$ and $e''$, then this relation should not contradict the other two 
relations. For example, if $e~>~e'$ and $e'~>~e''$, then it should be 
impossible for a well-formed event-string to also have the relation $e~<~e''$, 
as this would break the interval-like fluent constraint mentioned above.

These last two points are interesting in particular, as they lead to a specific 
kind of superposition between strings $s, s' \in \Sigma^+$ when some 
symbol $\alpha \in s$ is equal to some other symbol $\alpha' \in s'$. In this 
scenario, the symbols must unify when superposing the strings, in order to 
create a well-formed event-string in accordance with the above two constraints. 
To achieve this, when a symbol $\alpha$ in $s$ is also present in $s'$, and the 
asynchronous superposition of these strings is desired, padding is carried out 
as normal, but superposition is only permitted of those results of padding 
in which the indices of the matching symbols are equal. To do otherwise would 
permit event-strings which are not well-formed.

\cite{allen1983maintaining} gives a transitivity table showing the inferred 
possible relations between two events $a$ and $c$, given the relation between 
each and an intermediary event, $b$. Each cell of the table shows simply the 
symbol which represent the binary relation -- we may improve on the readability 
of this by showing explicitly the well-formed event-string(s) formed by the 
asynchronous superposition in each case. A fragment of the entire table is 
shown in Table \ref{table:transitivity} below:

\setlength{\tabcolsep}{2.5pt}
\renewcommand{\arraystretch}{3}
\noindent
\begin{table}[h!]
\centering
\begin{tabular}{| c | c | c | c | c |}
	\hline
	& 
	\thead{\textbf{``before"}\\\ebox{}\ebox{b}\ebox{}\ebox{c}\ebox{}} & 
	\thead{\textbf{``during"}\\\ebox{}\ebox{c}\ebox{b,c}\ebox{c}\ebox{}} & 
	\thead{\textbf{``meets"}\\\ebox{}\ebox{b}\ebox{c}\ebox{}} & 
	\thead{\textbf{``starts"}\\\ebox{}\ebox{b,c}\ebox{c}\ebox{}}\\
	\hline
	\thead{\textbf{``before"}\\\ebox{}\ebox{a}\ebox{}\ebox{b}\ebox{}} & 
	\thead{\ebox{}\ebox{a}\ebox{}\ebox{b}\ebox{}\ebox{c}\ebox{}} & 
	\thead{\ebox{}\ebox{a}\ebox{}\ebox{c}\ebox{b,c}\ebox{c}\ebox{},\\ 
		\ebox{}\ebox{a}\ebox{a,c}\ebox{c}\ebox{b,c}\ebox{c}\ebox{},\\ 
		\ebox{}\ebox{a}\ebox{c}\ebox{b,c}\ebox{c}\ebox{},\\ 
		\ebox{}\ebox{c}\ebox{a,c}\ebox{c}\ebox{b,c}\ebox{c}\ebox{},\\ 
		\ebox{}\ebox{a,c}\ebox{c}\ebox{b,c}\ebox{c}\ebox{}} & 
	\thead{\ebox{}\ebox{a}\ebox{}\ebox{b}\ebox{c}\ebox{}} & 
	\thead{\ebox{}\ebox{a}\ebox{}\ebox{b,c}\ebox{c}\ebox{}}\\
	\hline
	\thead{\textbf{``during"}\\\ebox{}\ebox{b}\ebox{a,b}\ebox{b}\ebox{}} & 
	\thead{\ebox{}\ebox{b}\ebox{a,b}\ebox{b}\ebox{}\ebox{c}\ebox{}} & 
	\thead{\ebox{}\ebox{c}\ebox{b,c}\ebox{a,b,c}\ebox{b,c}\ebox{c}\ebox{}} & 
	\thead{\ebox{}\ebox{b}\ebox{a,b}\ebox{b}\ebox{c}\ebox{}} & 
	\thead{\ebox{}\ebox{b,c}\ebox{a,b,c}\ebox{b,c}\ebox{c}\ebox{}}\\
	\hline
	\thead{\textbf{``meets"}\\\ebox{}\ebox{a}\ebox{b}\ebox{}} & 
	\thead{\ebox{}\ebox{a}\ebox{b}\ebox{}\ebox{c}\ebox{}} & 
	\thead{\ebox{}\ebox{a}\ebox{a,c}\ebox{b,c}\ebox{c}\ebox{},\\
		\ebox{}\ebox{c}\ebox{a,c}\ebox{b,c}\ebox{c}\ebox{},\\
		\ebox{}\ebox{a,c}\ebox{b,c}\ebox{c}\ebox{}} &
	\thead{\ebox{}\ebox{a}\ebox{b}\ebox{c}\ebox{}} & 
	\thead{\ebox{}\ebox{a}\ebox{b,c}\ebox{c}\ebox{}}\\
	\hline
	\thead{\textbf{``starts"}\\\ebox{}\ebox{a,b}\ebox{b}\ebox{}} & 
	\thead{\ebox{}\ebox{a,b}\ebox{b}\ebox{}\ebox{c}\ebox{}} & 
	\thead{\ebox{}\ebox{c}\ebox{a,b,c}\ebox{b,c}\ebox{c}\ebox{}} & 
	\thead{\ebox{}\ebox{a,b}\ebox{b}\ebox{c}\ebox{}} & 
	\thead{\ebox{}\ebox{a,b,c}\ebox{b,c}\ebox{c}\ebox{}} \\
	\hline
\end{tabular}
\caption{Fragment of Allen Transitivity Table using event-strings}
\label{table:transitivity}
\end{table}

\noindent
Here and in the original table, only three events are mentioned: $a$, $b$, and 
$c$. We can see that the asynchronous superposition of an event-string 
$s_{a,b}$ mentioning $a$ and $b$ with an event-string $s_{b,c}$ mentioning $b$ 
and $c$ gives a language $L$ of event-strings mentioning all three events. 
Applying the reduct $\rho_{\{a,b\}}$ to any string in $L$ (and 
block compressing the result) should give back exactly $s_{a,b}$, and likewise 
applying $\rho_{\{b,c\}}$ to any string in $L$ should give back exactly 
$s_{b,c}$. It should, in theory, be possible to generalise this to any number 
of events, ensuring the same level of readability by using event-strings. Then 
for an event-string of arbitrary length featuring any number of fluents, we can 
apply the reduct $\rho_{\{e,e'\}}$ and block compress the result to obtain the 
Allen Relation between fluents $e$ and $e'$ by comparing with the event-strings 
corresponding to each Allen Relation, laid out in Section \ref{event rep}.

Finally, we may also introduce further constraints if external 
information is 
available, and these might be simply intersected with the result of a 
superposition: $(s\ \&_* \ s') ~\cap~ C$, where $C$ represents the constraints 
to be applied, for example, ``$e$ is among the first events to occur in the 
string $s$" (true iff $s ~\&~ $\eboxb{}\eboxb{e}\eboxb{}$^{^{\mbox{$\ast$}}} = 
s$). This allows for extension beyond Allen Relations in the future.

\section{Application to TLINKs}\label{application}
As mentioned in Section \ref{event rep}, attempting to 
generate all of the possible worlds becomes difficult when using just the 
unconstrained bounded event-strings alone, as there are just too many (rarely, 
if ever, featuring five fluents or fewer). Instead, we begin by looking at just 
those fluents which are linked to another by Allen Relation, which we may 
extract from the TLINKs in a document annotated with ISO-TimeML, as noted in 
Section \ref{motivation}.

As each relation corresponds exactly to one possible model, we translate the 
TLINKs immediately to the appropriate event-strings, and superpose these 
according to the constraints mentioned in Section \ref{constraints}. This 
allows us to avoid simply superposing based on the fluents, and bypasses having 
to generate the initial 13 possibilities. In this way, we may generate a much 
smaller set of possible outcomes from a larger number of bounded events.

We can now rewrite the information given in the TLINKs 
(\ref{example:tlink1})--(\ref{example:tlink3}) as the following 
event-strings:\footnote{Using https://www.scss.tcd.ie/$\sim$dwoods/timeml/ to 
quickly extract TLINKs from an ISO-TimeML document and translate them to 
event-strings. A non-trivial extension of this program which computes results 
of superposition is in the works.}

\begin{subequations}
\begin{align}
\ebox{}\ebox{t1}\ebox{ei1,t1}\ebox{t1}\ebox{}\\
\ebox{}\ebox{ei9}\ebox{t1,ei9}\ebox{ei9}\ebox{}\\
\ebox{}\ebox{ei9}\ebox{}\ebox{ei10}\ebox{}
\end{align}
\end{subequations}

We may asynchronously superpose these event-strings while respecting the 
established constraints in order to generate a new event-string which contains 
all of the information from each of the inputs:
\begin{align}
\ebox{}\ebox{ei9}\ebox{t1,ei9}\ebox{ei1,t1,ei9}\ebox{t1,ei9}\ebox{ei9}\ebox{}\ebox{ei10}\ebox{}
\end{align}
A clear advantage here is the compact, readable representation. Furthermore, we 
can infer the Allen Relation between any two fluents in this event-string $s$ 
by applying a reduct and block compressing the result. For example, to infer the
relationship between $ei1$ and $ei10$, we obtain $s' = 
\bc{}(\rho_{\{ei1,ei10\}}(s)) = $ \ebox{}\ebox{ei1}\ebox{}\ebox{ei10}\ebox{}, 
from which we can conclude that the Allen Relation $ei1$ \textit{before} $ei10$ 
holds.

A drawback here is that for this to be effective, it relies on the events being 
heavily constrained by their interrelations. If there are too few TLINKs 
relative to the number of events, we still run into the problem of 
combinatorial explosion. What's more, as seen in Table 
\ref{table:transitivity}, some asynchronous superpositions will still generate 
multiple disjoint possibilities (such as $a$ \textit{before} $b$ with $b$ 
\textit{during} $c$), which will also impact the combinatorial problem.

An additional issue in computation of the superposition of events arises as 
multiple superposition operations are be carried out in sequence, 
meaning unordered data may lead to a much less efficient calculation of final 
results. For example, 
\eboxl{}\eboxl{a}\eboxl{}\eboxl{b}\eboxl{}
 $\&_*$ 
 \eboxl{}\eboxl{b}\eboxl{}\eboxl{c}\eboxl{}
 $\&_*$ 
 \eboxl{}\eboxl{c}\eboxl{}\eboxl{d}\eboxl{}
  and 
  \eboxl{}\eboxl{a}\eboxl{}\eboxl{b}\eboxl{}
  $\&_*$ 
  \eboxl{}\eboxl{c}\eboxl{}\eboxl{d}\eboxl{}
  $\&_*$ 
  \eboxl{}\eboxl{b}\eboxl{}\eboxl{c}\eboxl{}
  should produce the same, single output: 
 \eboxl{}\eboxl{a}\eboxl{}\eboxl{b}\eboxl{}\eboxl{c}\eboxl{}\eboxl{d}\eboxl{},
  which they do. However, due to the respective orderings, the first sequence 
 will arrive at that conclusion much faster as 
 \eboxl{}\eboxl{a}\eboxl{}\eboxl{b}\eboxl{}
 $\&_*$ 
 \eboxl{}\eboxl{b}\eboxl{}\eboxl{c}\eboxl{}
  has one possible outcome 
  (\eboxl{}\eboxl{a}\eboxl{}\eboxl{b}\eboxl{}\eboxl{c}%
 \eboxl{}), which can immediately be asynchronously superposed with 
 \eboxl{}\eboxl{c}\eboxl{}\eboxl{d}\eboxl{}
  to produce the final output. However, 
  \eboxl{}\eboxl{a}\eboxl{}\eboxl{b}\eboxl{}
  $\&_*$ 
  \eboxl{}\eboxl{c}\eboxl{}\eboxl{d}\eboxl{}
   has 321 possible outcomes, each of which must be individually asynchronously 
  superposed with 
  \eboxl{}\eboxl{b}\eboxl{}\eboxl{c}\eboxl{},
   only to come to the same conclusion, as only one of these results is 
  well-formed.

One potential way to work around this pitfall is a grouping and ordering stage, 
where initially only events linked by some relation may be superposed, and only 
after the operand strings have been sorted to some optimal order, whereby the 
event-strings with the most shared fluents are grouped. It may be 
prudent to only perform superposition at all on event-strings which may be 
linked through one or more relations or shared fluents. In this way, new, 
underspecified events may be formed from the output strings. Consider the 
scenario with $e_1, \ldots, e_8 \in A$, and the following Allen 
Relations:
\begin{subequations}
\begin{align}
e_1 <& ~e_4\\
e_1 ~\mathbf{m}& ~e_2\\
e_2 ~\mathbf{di}& ~e_3\\
e_5 ~\mathbf{s}& ~e_7\\
e_8 >& ~e_5
\end{align}
\end{subequations}
Let us cluster the fluents as follows: for each fluent $a \in A$, fix a set $P 
= \{ a \}$, and a set $S$ whose members are these sets $P$.
\begin{align}
S = \{\{e_1\},\{e_2\},\{e_3\},\{e_4\},\{e_5\},\{e_6\},\{e_7\},\{e_8\}\}
\end{align}
Next, check for an Allen Relation between each pair of fluents $a$ and $a'$. If 
a relation exists, add the fluent $a'$ to the set $P$.
\begin{align}
S' = \{\{e_1, e_4, e_2\},\{e_2, e_3\}, \{e_3\}, \{e_4\}, \{e_5, e_7\}, \{e_6\}, 
\{e_7\}, \{e_8, e_5\}\}
\end{align}
Finally, for each pair of sets $P, Q$, if $|P ~\cap~ Q| > 0$, form $R = P 
~\cup~Q$, adding $R$ to the set $S''$ and discarding $P, Q$. Add the remaining 
sets 
from $S'$ to $S''$.
\begin{align}
S'' = \{\{e_1, e_4, e_2, e_3\}, \{e_5, e_7, e_8\}, \{e_6\}\}
\end{align}
We might form the underspecified event groups $E_1$ and 
$E_2$ to refer to these first two clusters, at which point we may freely treat 
these groups as normal bounded events, and perform asynchronous superposition 
on their event-strings, as well as with that of $e_6$ -- reducing the 
number of inputs from 8 to 3.

Additionally, various weightings might be considered as a method of 
priority-ordering in the case of a large $A$, such as the number of 
component events in an underspecified event group, or the number of relations 
linking to a particular event.
\section{Conclusion}\label{conclusion}
We have explored in this work the possibility of using strings as models for 
event data, motivated by their nature as computational entities. The 
operation of asynchronous superposition was described for composing strings 
which represent finite, bounded events, as well as its limits in terms of 
massive growth in the number of outputs when the operation is repeated in 
sequence. The problem is addressed by 
constraining the strings which may be superposed, with the 13 
Allen Relations forming the main part of these, as these can be found in 
annotated corpora such as TIMEBANK, using the ISO-TimeML standard.

Future work on this topic will further develop the constraints on asynchronous 
superposition and expand on the features of the current framework, while also 
examining the use of alternative models to approach the same issue, such as 
using finite state automata, or a hybrid string/FSA approach. We will 
additionally explore the potential of employing methods from distributed 
computing in order to tackle the combinatorial explosion that occurs in 
asynchronously superposing unconstrained bounded events.
%%%%%%%%%%%%%%%%%%%%%%%%%%%%%%%%%%%%%%%%%%%%%%%%%%%%%%%%%%%%%%%%%%%%%%%%%%%%%%%%
%                                                                              %
%                             Actual Document Ends                             %
%                                                                              %
%%%%%%%%%%%%%%%%%%%%%%%%%%%%%%%%%%%%%%%%%%%%%%%%%%%%%%%%%%%%%%%%%%%%%%%%%%%%%%%%
\section*{Acknowledgements}
This research is supported by Science Foundation Ireland (SFI) through the CNGL 
Programme (Grant 12/CE/I2267) in the ADAPT Centre 
(\url{https://www.adaptcentre.ie}) at Trinity College Dublin. The
ADAPT Centre for Digital Content Technology is funded under the SFI Research 
Centres Programme (Grant 13/RC/2106) and is co-funded under the European 
Regional Development Fund.

\bibliographystyle{chicago}{}
\bibliography{refs}
\end{document}